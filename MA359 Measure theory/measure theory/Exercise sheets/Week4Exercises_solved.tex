\documentclass[11pt]{article}

\usepackage{mathtools}
\usepackage{amsmath, amsthm, amsfonts,amssymb}
\usepackage{enumitem}
\usepackage{graphicx}
\usepackage{colortbl}
\usepackage{tikz}
\usepackage[utf8]{inputenc}
\usepackage{esint}
\usepackage{mathrsfs}
\usepackage{subfig,float}
\usepackage[T1]{fontenc}
\usepackage{mathrsfs}  
\usepackage{bbm} 
\usepackage{enumitem}
\usepackage{enumerate}
\usepackage{mathtools}
\usepackage{dsfont}
\newcommand{\set}[1]{\left\{#1\right\}}

\def\grad{\nabla}
\DeclareMathOperator{\dive}{div}
\DeclareMathOperator{\supp}{supp}
\DeclareMathOperator{\essup}{ess\,sup}
\DeclareMathOperator{\Lip}{Lip}
\DeclareMathOperator{\sgn}{sgn}

% Notation for differentials
\def\d{\,\mathrm{d}}
\def\dv{\d v}
\def \ddt{\frac{\mathrm{d}}{\mathrm{d}t}}
\def \ddt{\frac{\mathrm{d}}{\mathrm{d}t}}
\def \ddr{\frac{\mathrm{d}}{\mathrm{d}r}}
\newcommand{\sign}{\text{sign}}
\DeclareMathOperator{\hess}{Hess}

% Generate a PDF with hyperlinks in references.
\usepackage[colorlinks=true,linkcolor=blue,citecolor=blue,urlcolor=blue,breaklinks]{hyperref}

% Bibliography
%-----------------------------------------------------------------

% This uses a bibliography style which hyperlinks the paper titles to
% the paper URL specified in the bibtex file. It also uses natbib,
% which cites papers by name such as Euler (1770) instead of [17].
\usepackage{hyperref}
\usepackage{breakurl}
\usepackage[square,sort,comma,numbers]{natbib}
%\usepackage{natbib}
\usepackage{url}
%\bibliographystyle{plainnat-linked}
\bibliographystyle{plain}

%\usepackage[notcite,notref]{showkeys}
%\usepackage{hyperref}
%\usepackage{breakurl}
\usepackage[square,sort,comma,numbers]{natbib}
%\usepackage{natbib}
%\usepackage{url}
%\usepackage[colorlinks=blue,linkcolor=blue,citecolor=blue,urlcolor=blue,breaklinks]{hyperref}
\bibliographystyle{plain}
\DeclarePairedDelimiter\abs{\lvert}{\rvert}%
\DeclarePairedDelimiter\norm{\lVert}{\rVert}%


\addtolength{\oddsidemargin}{-.875in}
\addtolength{\evensidemargin}{-.875in}
\addtolength{\textwidth}{1.75in}

\addtolength{\topmargin}{-.875in}
\addtolength{\textheight}{1.75in}

% Shortcuts
%-----------------------------------------------------------------

%\newcommand{\abs}[1]{\left\mid#1\right\mid}
%\newcommand{\ap}[1]{\left\langle#1\right\rangle}
%\newcommand{\norm}[1]{\left\mid#1\right\mid}
% \newcommand{\tnorm}[1]{\left\mid\!\left\mid\!\left\mid#1\right\mid\!\right\mid\!\right\mid}

\definecolor{lpink}{rgb}{0.96, 0.76, 0.76}
\definecolor{dpink}{rgb}{0.97, 0.51, 0.47}
\definecolor{sky}{rgb}{0.53, 0.81, 0.92}
\definecolor{salmon}{rgb}{1.0, 0.55, 0.41}
\definecolor{orman}{rgb}{0.24, 0.7, 0.44}
\definecolor{aciksari}{rgb}{0.91, 0.84, 0.42}
\definecolor{dgrey}{rgb}{0.52, 0.52, 0.51}

\def\R{\mathbb{R}}
\def\C{\mathbb{C}}
\def\P{\mathscr{P}}
\def\NN{\mathbb{N}}
\def\Q{\mathbb{Q}}

\def\ird{\int_{\mathbb{R}^N}}
\def\d{\,\mathrm{d}}
\def\dx{\,\mathrm{d}x}
\def\dy{\,\mathrm{d}y}
\def\p{\,\partial}
\newcommand{\en}{\mathcal{H}}
\newcommand{\havva}[1]{{\textcolor{blue}{[\textbf{H:} #1]}}}

% Operators

\def\grad{\nabla}
\def\weakto{\rightharpoonup}

\DeclareMathOperator{\divergence}{div}
%\newcommand{\dv}[1]{\divergence \left(#1\right)}
%
%\DeclareMathOperator{\supp}{supp}
%\DeclareMathOperator{\essup}{ess\,sup}
%\DeclareMathOperator{\Lip}{Lip}
\DeclareMathOperator{\law}{law}

\DeclareMathOperator{\Pf}{Pf.}
\DeclareMathOperator{\Fp}{Fp.}
\DeclareMathOperator{\pv}{pv.}

\newcommand{\for}{\quad \text{ for all }}
\newcommand{\tb}[1]{\textcolor{blue}{#1}}


\newcommand{\OmT}{\Omega\times (0,T)}
\let\pa\partial
\let\na\nabla
\newcommand{\red}[1]{\textcolor{red}{#1}}
\newcommand{\cD}{\mathcal{D}}
\let\eps\varepsilon
\newcommand{\wen}{w^{(\varepsilon,N)}}
\newcommand{\OmTc}{\overline{\Omega}\times [0,T]}
\newcommand{\rrhoA}{\sqrt{\rho_A}}
\newcommand{\rrhoB}{\sqrt{\rho_B}}
\newcommand{\rrhoAB}{\sqrt{\rho_A\rho_B}}

% Theorems
%-----------------------------------------------------------------
\newtheorem{thm}{Theorem}[section]
% \newtheorem{thm}{Theorem}
\newtheorem{cor}[thm]{Corollary}
\newtheorem{lem}[thm]{Lemma}
\newtheorem{prp}[thm]{Proposition}
\newtheorem{claim}[thm]{Claim}
% \newtheorem{hyp}[thm]{Hypothesis}
\newtheorem{hyp}{Hypothesis}
\theoremstyle{definition}
\newtheorem{dfn}[thm]{Definition}
\theoremstyle{remark}
\newtheorem{remark}[thm]{Remark}
\newtheorem{ex}[thm]{Example}
\newtheorem{q}{Question}
\newenvironment{ans}{\paragraph{Answer:}}{\hfill$\square$ \vspace{10pt}}

\hypersetup{pdftitle={Measure Theory}}
\hypersetup{pdfauthor={Josephine Evans }}


\author{
Josephine Evans
}
\title{Measure Theory: Exercises (not for credit)}
%\date{}

\makeindex

\begin{document}
\maketitle
This is a shorter sheet because last week was longer and you have the assignment as well. I think the most useful question is 3.

\begin{q}
Let $A_n$ be a sequence of measurable sets. Show that $1_{A_n}$ converges to $0$ in measure if and only if $\mu(A_n) \rightarrow 0$. Furthermore show that $1_{A_n}$ converges almost everywhere if and only if $\mu(\bigcap_{n=1}^\infty \bigcup_{k=n}^\infty A_k) = 0$.
\end{q}
\begin{ans}
This is mainly an exercise in understanding all the definitions. The first result is almost by definition. $1_{A_n}$ tends to $0$ in measure if for every $\epsilon>0$ the measure of $\{x \,:\, 1_{A_n} > \epsilon\}$ tends to $0$. When $\epsilon >1$ this set is empty so the result is true. When $\epsilon \leq 1$ we have $\{x \,:\, 1_{A_n} > \epsilon\} = A_n$ so convergence happens in measure iff $\mu(A_n) \rightarrow 0$.

The second part is a bit more subtle. $1_{A_n}$ tends to $0$ at $x$ if $x$ does not appear in infinitel many $A_n$. The set of points which appears in infinitely many $A_n$ is $\bigcap_n \bigcup_{m \geq n} A_m$ so the $1_{A_n}$ converges in measure if and only if $\mu(\bigcap_n \bigcup_{m \geq n} A_m) = 0$.
\end{ans}

\begin{q}
Let $(f_n)_{n \geq 1}, f, g$ be Borel measurable functions from $\mathbb{R} \rightarrow \mathbb{R}$. Suppose further that $g$ is continuous. Show that if $f_n \rightarrow f$ almost everywhere then $g\circ f_n \rightarrow g \circ f$ almost everywhere. Can the conclusion fail if $g$ is only continuous almost everywhere. 
\end{q}
\begin{ans}
As $f_n$ converges to $f$ almost everywhere there is a set $A$ with $f_n(x) \rightarrow f(x)$ for every $x \in A$ and $\lambda(A^c) = 0$. Now if $x \in A$, since $g$ is continuous $g(f_n(x)) \rightarrow g(f(x))$ so $g \circ f_n \rightarrow g \circ f$ almost everywhere. 

Now let $f_n(x) = x^n$ for $x \in (0,1)$ and $1$ everywhere else. Then $f_n$ converges to $0$ almost everywhere. Let $g(x) = 1$ when $x \neq 0$ but $g(0) = 0$. Then $g(f_n(x)) = 1$ for every $x \neq 0$ but $g(f(x))=0$ for every $x$ so $g \circ f_n$ does not converge to $g \circ f$ almost everywhere.
\end{ans}

\begin{q}
Suppose that $f$ is a measurable function from $(E, \mathcal{E}, \mu)$ to $(F, \mathcal{F})$ show that the image measure defined by $\nu(A) = \mu(f^{-1}(A))$ for all $A \in \mathcal{F}$ is indeed a measure.
\end{q}
\begin{ans}
We need to check that $\nu$ satisfies all the axioms to be a measure. $\nu(\emptyset) = \mu(f^{-1}(\emptyset)) = \mu(\emptyset) = 0$. We also have that $\nu \geq 0$ since $\mu \geq 0$. We now need to show that $\mu$ is countably additive. Suppose $A_n$ is a sequence of disjoint sets in $\mathcal{F}$ then $\mu(\bigcup_n A_n) = \mu(f^{-1}(\bigcup_n A_n)) = \mu ( \bigcup_n f^{-1}(A_n))$ now the fact that the $A_n$ are disjoint imples that $f^{-1}(A_n)$ are disjoint ($x$ can't end up in two disjoint sets). So by countable additivity of $\mu$ we have $\mu( \bigcup_n f^{-1}(A_n))= \sum_n \mu(f^{-1}(A_n)) = \sum_n \nu (A_n)$. Therefore $\nu$ is countably additive. We have then shown that $\nu$ is a measure.
\end{ans}

\begin{q}
Suppose that $(E, \mathcal{E}, \mu)$ is a $\sigma$-finite measure space and $f_n$ is a sequence of real valued measurable functions on $E$. Suppose that $f_n \rightarrow f$ almost everywhere. Show that there exists a sequence of sets $A_n$ and a set $B$ such that $E = \bigcup_n A_n \cup B$ and $\mu(B) =0$ and $f_n$ converges uniformly on each $A_n$.
\end{q}
\begin{ans}
I think this is a hard questions. We need to use Egoroff's theorem. First by $\sigma$-finiteness there exists a sequence of sets $E_k$ such that $\mu(E_k) < \infty$ for every $k$ and $E= \bigcup_k E_k$. Then let us fix $\epsilon$ we are going to show an intermediate claim: Given $\epsilon >0$ there exists a sequence of sets $A_{k, \epsilon}$ such that $f_n$ converges uniformly on each $A_{k,\epsilon}$ and $\mu((\bigcup_k A_{k, \epsilon})^c) < \epsilon$. In order to do this we apply Egoroff's theorem to each $E_k$ to find a set $A_k \subseteq E_k$ with $\mu(E_k \setminus A_k) < \epsilon 2^{-k}$. Then we have that $f_n$ converges uniformly on each $A_{k, \epsilon}$ and $\mu(E \setminus \bigcup_k A_{k,\epsilon}) = \mu( \bigcup_k (E_k \setminus A_{k, \epsilon}) \leq \sum_k \mu(E_k \setminus A_{k,\epsilon}) \leq \epsilon \sum_k 2^{-k} = \epsilon$. We have then proved the claim for each $\epsilon$.

Now we take the sets $A_{k, 2^{-m}}$ these form a countable collection what we really need to do is reorder them but lets show the theorem in an explicit way. Let $A_{n} = \bigcup_{k, m \leq n} A_{k, 2^{-m}}$ then since $f_n$ converges uniformly on each $A_{k, 2^{-m}}$ it converges uniformly on any finite union of such sets so on each $A_n$. So then look at $\mu(E \setminus \bigcup_n A_n) \leq \mu(E \setminus \bigcup_k A_{k, 2^{-m}})$ for any $m$ by monotonicity of $\mu$. Therefore $\mu(E \setminus \bigcup_n A_n) \leq 2^{-m}$ for any $m$, hence $\mu(E \setminus \bigcup_n A_n) = 0$. Therefore set $B = E \setminus \bigcup_n A_n)$ and we have proved the result.
\end{ans}
\end{document}