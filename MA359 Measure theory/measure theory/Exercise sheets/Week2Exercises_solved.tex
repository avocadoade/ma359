\documentclass[11pt]{article}

\usepackage{mathtools}
\usepackage{amsmath, amsthm, amsfonts,amssymb}
\usepackage{enumitem}
\usepackage{graphicx}
\usepackage{colortbl}
\usepackage{tikz}
\usepackage[utf8]{inputenc}
\usepackage{esint}
\usepackage{mathrsfs}
\usepackage{subfig,float}
\usepackage[T1]{fontenc}
\usepackage{mathrsfs}  
\usepackage{bbm} 
\usepackage{enumitem}
\usepackage{enumerate}
\usepackage{mathtools}
\usepackage{dsfont}
\newcommand{\set}[1]{\left\{#1\right\}}

\def\grad{\nabla}
\DeclareMathOperator{\dive}{div}
\DeclareMathOperator{\supp}{supp}
\DeclareMathOperator{\essup}{ess\,sup}
\DeclareMathOperator{\Lip}{Lip}
\DeclareMathOperator{\sgn}{sgn}

% Notation for differentials
\def\d{\,\mathrm{d}}
\def\dv{\d v}
\def \ddt{\frac{\mathrm{d}}{\mathrm{d}t}}
\def \ddt{\frac{\mathrm{d}}{\mathrm{d}t}}
\def \ddr{\frac{\mathrm{d}}{\mathrm{d}r}}
\newcommand{\sign}{\text{sign}}
\DeclareMathOperator{\hess}{Hess}

% Generate a PDF with hyperlinks in references.
\usepackage[colorlinks=true,linkcolor=blue,citecolor=blue,urlcolor=blue,breaklinks]{hyperref}

% Bibliography
%-----------------------------------------------------------------

% This uses a bibliography style which hyperlinks the paper titles to
% the paper URL specified in the bibtex file. It also uses natbib,
% which cites papers by name such as Euler (1770) instead of [17].
\usepackage{hyperref}
\usepackage{breakurl}
\usepackage[square,sort,comma,numbers]{natbib}
%\usepackage{natbib}
\usepackage{url}
%\bibliographystyle{plainnat-linked}
\bibliographystyle{plain}

%\usepackage[notcite,notref]{showkeys}
%\usepackage{hyperref}
%\usepackage{breakurl}
\usepackage[square,sort,comma,numbers]{natbib}
%\usepackage{natbib}
%\usepackage{url}
%\usepackage[colorlinks=blue,linkcolor=blue,citecolor=blue,urlcolor=blue,breaklinks]{hyperref}
\bibliographystyle{plain}
\DeclarePairedDelimiter\abs{\lvert}{\rvert}%
\DeclarePairedDelimiter\norm{\lVert}{\rVert}%


\addtolength{\oddsidemargin}{-.875in}
\addtolength{\evensidemargin}{-.875in}
\addtolength{\textwidth}{1.75in}

\addtolength{\topmargin}{-.875in}
\addtolength{\textheight}{1.75in}

% Shortcuts
%-----------------------------------------------------------------

%\newcommand{\abs}[1]{\left\mid#1\right\mid}
%\newcommand{\ap}[1]{\left\langle#1\right\rangle}
%\newcommand{\norm}[1]{\left\mid#1\right\mid}
% \newcommand{\tnorm}[1]{\left\mid\!\left\mid\!\left\mid#1\right\mid\!\right\mid\!\right\mid}

\definecolor{lpink}{rgb}{0.96, 0.76, 0.76}
\definecolor{dpink}{rgb}{0.97, 0.51, 0.47}
\definecolor{sky}{rgb}{0.53, 0.81, 0.92}
\definecolor{salmon}{rgb}{1.0, 0.55, 0.41}
\definecolor{orman}{rgb}{0.24, 0.7, 0.44}
\definecolor{aciksari}{rgb}{0.91, 0.84, 0.42}
\definecolor{dgrey}{rgb}{0.52, 0.52, 0.51}

\def\R{\mathbb{R}}
\def\C{\mathbb{C}}
\def\P{\mathscr{P}}
\def\NN{\mathbb{N}}
\def\Q{\mathbb{Q}}

\def\ird{\int_{\mathbb{R}^N}}
\def\d{\,\mathrm{d}}
\def\dx{\,\mathrm{d}x}
\def\dy{\,\mathrm{d}y}
\def\p{\,\partial}
\newcommand{\en}{\mathcal{H}}
\newcommand{\havva}[1]{{\textcolor{blue}{[\textbf{H:} #1]}}}

% Operators

\def\grad{\nabla}
\def\weakto{\rightharpoonup}

\DeclareMathOperator{\divergence}{div}
%\newcommand{\dv}[1]{\divergence \left(#1\right)}
%
%\DeclareMathOperator{\supp}{supp}
%\DeclareMathOperator{\essup}{ess\,sup}
%\DeclareMathOperator{\Lip}{Lip}
\DeclareMathOperator{\law}{law}

\DeclareMathOperator{\Pf}{Pf.}
\DeclareMathOperator{\Fp}{Fp.}
\DeclareMathOperator{\pv}{pv.}

\newcommand{\for}{\quad \text{ for all }}
\newcommand{\tb}[1]{\textcolor{blue}{#1}}


\newcommand{\OmT}{\Omega\times (0,T)}
\let\pa\partial
\let\na\nabla
\newcommand{\red}[1]{\textcolor{red}{#1}}
\newcommand{\cD}{\mathcal{D}}
\let\eps\varepsilon
\newcommand{\wen}{w^{(\varepsilon,N)}}
\newcommand{\OmTc}{\overline{\Omega}\times [0,T]}
\newcommand{\rrhoA}{\sqrt{\rho_A}}
\newcommand{\rrhoB}{\sqrt{\rho_B}}
\newcommand{\rrhoAB}{\sqrt{\rho_A\rho_B}}

% Theorems
%-----------------------------------------------------------------
\newtheorem{thm}{Theorem}[section]
% \newtheorem{thm}{Theorem}
\newtheorem{cor}[thm]{Corollary}
\newtheorem{lem}[thm]{Lemma}
\newtheorem{prp}[thm]{Proposition}
\newtheorem{claim}[thm]{Claim}
% \newtheorem{hyp}[thm]{Hypothesis}
\newtheorem{hyp}{Hypothesis}
\theoremstyle{definition}
\newtheorem{dfn}[thm]{Definition}
\theoremstyle{remark}
\newtheorem{remark}[thm]{Remark}
\newtheorem{ex}[thm]{Example}
\newtheorem{q}{Question}
\newenvironment{ans}{\paragraph{Answer:}}{\hfill$\square$ \vspace{20pt}}

\hypersetup{pdftitle={Measure Theory}}
\hypersetup{pdfauthor={Josephine Evans }}


\author{
Josephine Evans
}
\title{Measure Theory: Exercises (not for credit)}
%\date{}

\makeindex

\begin{document}
\maketitle

\begin{q}
Let $C$ be a countable subset of $\mathbb{R}$. Show that $\lambda^*(C) = 0$.
\end{q}
\begin{ans}
We first show that $\lambda^*(\{b\}) = 0$. We have $\lambda^*(\{b\}) \leq \lambda((b-1/n,b]) = 1/n$. So letting $n \rightarrow \infty$ gives $\lambda(\{b\})=0$. Then we can write $C = \{b_1\} \cup \{b_2\} \cup \{b_3\} \cup \dots$ then, by countable subadditivity, we have $\lambda^*(C) \leq \sum_n \lambda^*(\{b_n\}) = 0$.
\end{ans}

\begin{q}
For each set $A \in \mathbb{R}^d$ show that there is a Borel subset, $B$, of $\mathbb{R}$ such that $\lambda(B) = \lambda^*(A)$, and $A \subseteq B$.
\end{q}
\begin{ans}
Let us define a sequence of intervals $I_{n,k}$ with two indices, by finding such a sequence with $A \subseteq \bigcup_k I_{n,k}$ and $\sum_k \lambda(I_{n,k}) \leq \lambda^*(A) + 2^{-n}$. Then write $J_n = \bigcup_k I_{n,k}$ for each $k$ and $B_n = \bigcap_{i=1}^n J_n$. Since the $I_{n,k}$ are Borel sets it follows that the $J_n$ are Borel sets and then that the $B_n$ are Borel sets. We also have $A \subseteq B_n$ for each $n$ and $\lambda(B_n) \leq \lambda(J_n) \leq \sum_k \lambda(I_{n,k}) \leq \lambda^*(A) + 2^{-n}$. We then let $n \rightarrow \infty$ and get $A \subseteq \bigcup_n B_n$ and $\lambda^*(\bigcup_n B_n) \leq \lambda^*(A)$. By monotonicity of $\lambda^*$ we also have $\lambda^*(A) \leq \lambda( \bigcup_n B_n)$ therefore $\lambda^*(A) = \lambda( \bigcup_n B_n)$ and $\bigcup_n B_n$ is a Borel set as it is the countable intersection of Borel sets.
\end{ans}

\begin{q}
Let $B$ be a Borel subset of $[0,1]$ show that there exists a finite, disjoint sequence of half open intervals $A$ such that $\lambda(A \triangle B) \leq \epsilon$. Here $A \triangle B = (A^c \cap B) \cup (A \cap B^c)$. 
\end{q}
\begin{ans}
This is similar to question 1. Let us take a sequence $I_n$ of half open intervals such that $B \subseteq \bigcup_n I_n$ and $\sum_n \lambda(I_n) \leq \lambda(B) + \epsilon/2$. Now since the sum $\sum_n \lambda(I_n)$ converges there exists an $N$ such that $\sum_{n \geq N} \lambda(I_n) < \epsilon/2$. We then write $A= \bigcup_{n=1}^{N-1}I_n$ this is a finite union of half open intervals so can be expressed as a finite disjoint union of half open intervals. We also have that $B \cap A^c \subseteq \bigcup_{n \geq N} I_n$ therefore $\lambda( B \cap A^c) \leq \sum_{n \geq N} \lambda(I_n) \leq \epsilon/2$. We also have that $A \cap B^c \subseteq \bigcup_{n=1}^\infty I_n \setminus B$ so $\lambda(A \cap B^c) \leq \lambda\left(\bigcup_{n=1}^\infty I_n \setminus B\right) = \lambda(\bigcup_n I_n) - \lambda(B) \leq \epsilon/2$. Putting this together gives $\lambda(A \triangle B) \leq \epsilon$.
\end{ans}


\begin{q}
Let $(E, \mathcal{E}, \mu)$ be a finite measure space and let $A_n$ be a sequence of measurable sets. Show that
\[ \mu \left( \bigcup_n \bigcap_{m \geq n} A_m \right) \leq \liminf_n \mu(A_n) \leq \limsup_n \mu(A_n) \leq \mu \left( \bigcap_n \bigcup_{m \geq n} A_m\right).\] Find an example to show that the last inequality is not necessarily true if $\mu$ is not finite.
\end{q}
\begin{ans}
The sequence $\bigcap_{m \geq n} A_m$ is increasing sequence. Therefore by continuity we have
\[ \mu\left( \bigcup_n \bigcap_{m \geq n}A_m\right) = \lim_n \mu(\bigcap_{m \geq n} A_m). \] By monotonocity of $\mu$ we have
\[ \mu(\bigcap_{m \geq n}A_m) \leq \mu(A_m), \quad \forall \, m \geq n \] therfore
\[ \mu(\bigcap_{m \geq n} A_m) \leq \inf_{m \geq n} \mu(A_m). \] Putting this all together gives the first inequality. 

The second inequality is just the fact that $\liminf \leq \limsup$. 

The sequence $\bigcup_{m \geq n} A_m$ is decreasing, and we are working in a finite measure space so all the sets have finite measure. By our continuity theorem this means that
\[ \mu \left( \bigcap_n \bigcup_{m \geq n} A_m\right) = \lim_n \mu( \bigcup_{m \geq n} A_m). \] By monotonicity we have
\[ \mu(A_m) \leq \mu \left( \bigcup_{m \geq n} A_m\right) \quad \forall \, m \geq n. \] Therefore we have
\[ \sup_{m \geq n} \mu(A_m) \leq \mu \left( \bigcup_{m \geq n} A_m\right). \] Putting all this together gives the last inequality. 

For a counterexample let $\mu = \lambda$ on $\mathbb{R}$ and  $A_n = [n,n+1]$ then $\bigcup_{m \geq n} A_m = [n, \infty)$ and $\bigcap_n \bigcup_{m \geq n} A_m = \bigcap_n [n,\infty) = \emptyset$. Therefore we have $\limsup_n \mu(A_n) = 1$ as $\mu(A_n) =1$ for every $n$, but $\mu( \bigcap_n \bigcup_{m\geq n}) = \mu(\emptyset) = 0$.
\end{ans}

\end{document}