\documentclass[11pt]{article}

\usepackage{mathtools}
\usepackage{amsmath, amsthm, amsfonts,amssymb}
\usepackage{enumitem}
\usepackage{graphicx}
\usepackage{colortbl}
\usepackage{tikz}
\usepackage[utf8]{inputenc}
\usepackage{esint}
\usepackage{mathrsfs}
\usepackage{subfig,float}
\usepackage[T1]{fontenc}
\usepackage{mathrsfs}  
\usepackage{bbm} 
\usepackage{enumitem}
\usepackage{enumerate}
\usepackage{mathtools}
\usepackage{dsfont}
\newcommand{\set}[1]{\left\{#1\right\}}

\def\grad{\nabla}
\DeclareMathOperator{\dive}{div}
\DeclareMathOperator{\supp}{supp}
\DeclareMathOperator{\essup}{ess\,sup}
\DeclareMathOperator{\Lip}{Lip}
\DeclareMathOperator{\sgn}{sgn}

% Notation for differentials
\def\d{\,\mathrm{d}}
\def\dv{\d v}
\def \ddt{\frac{\mathrm{d}}{\mathrm{d}t}}
\def \ddt{\frac{\mathrm{d}}{\mathrm{d}t}}
\def \ddr{\frac{\mathrm{d}}{\mathrm{d}r}}
\newcommand{\sign}{\text{sign}}
\DeclareMathOperator{\hess}{Hess}

% Generate a PDF with hyperlinks in references.
\usepackage[colorlinks=true,linkcolor=blue,citecolor=blue,urlcolor=blue,breaklinks]{hyperref}

% Bibliography
%-----------------------------------------------------------------

% This uses a bibliography style which hyperlinks the paper titles to
% the paper URL specified in the bibtex file. It also uses natbib,
% which cites papers by name such as Euler (1770) instead of [17].
\usepackage{hyperref}
\usepackage{breakurl}
\usepackage[square,sort,comma,numbers]{natbib}
%\usepackage{natbib}
\usepackage{url}
%\bibliographystyle{plainnat-linked}
\bibliographystyle{plain}

%\usepackage[notcite,notref]{showkeys}
%\usepackage{hyperref}
%\usepackage{breakurl}
\usepackage[square,sort,comma,numbers]{natbib}
%\usepackage{natbib}
%\usepackage{url}
%\usepackage[colorlinks=blue,linkcolor=blue,citecolor=blue,urlcolor=blue,breaklinks]{hyperref}
\bibliographystyle{plain}
\DeclarePairedDelimiter\abs{\lvert}{\rvert}%
\DeclarePairedDelimiter\norm{\lVert}{\rVert}%


\addtolength{\oddsidemargin}{-.875in}
\addtolength{\evensidemargin}{-.875in}
\addtolength{\textwidth}{1.75in}

\addtolength{\topmargin}{-.875in}
\addtolength{\textheight}{1.75in}

% Shortcuts
%-----------------------------------------------------------------

%\newcommand{\abs}[1]{\left\mid#1\right\mid}
%\newcommand{\ap}[1]{\left\langle#1\right\rangle}
%\newcommand{\norm}[1]{\left\mid#1\right\mid}
% \newcommand{\tnorm}[1]{\left\mid\!\left\mid\!\left\mid#1\right\mid\!\right\mid\!\right\mid}

\definecolor{lpink}{rgb}{0.96, 0.76, 0.76}
\definecolor{dpink}{rgb}{0.97, 0.51, 0.47}
\definecolor{sky}{rgb}{0.53, 0.81, 0.92}
\definecolor{salmon}{rgb}{1.0, 0.55, 0.41}
\definecolor{orman}{rgb}{0.24, 0.7, 0.44}
\definecolor{aciksari}{rgb}{0.91, 0.84, 0.42}
\definecolor{dgrey}{rgb}{0.52, 0.52, 0.51}

\def\R{\mathbb{R}}
\def\C{\mathbb{C}}
\def\P{\mathscr{P}}
\def\NN{\mathbb{N}}
\def\Q{\mathbb{Q}}

\def\ird{\int_{\mathbb{R}^N}}
\def\d{\,\mathrm{d}}
\def\dx{\,\mathrm{d}x}
\def\dy{\,\mathrm{d}y}
\def\p{\,\partial}
\newcommand{\en}{\mathcal{H}}
\newcommand{\havva}[1]{{\textcolor{blue}{[\textbf{H:} #1]}}}

% Operators

\def\grad{\nabla}
\def\weakto{\rightharpoonup}

\DeclareMathOperator{\divergence}{div}
%\newcommand{\dv}[1]{\divergence \left(#1\right)}
%
%\DeclareMathOperator{\supp}{supp}
%\DeclareMathOperator{\essup}{ess\,sup}
%\DeclareMathOperator{\Lip}{Lip}
\DeclareMathOperator{\law}{law}

\DeclareMathOperator{\Pf}{Pf.}
\DeclareMathOperator{\Fp}{Fp.}
\DeclareMathOperator{\pv}{pv.}

\newcommand{\for}{\quad \text{ for all }}
\newcommand{\tb}[1]{\textcolor{blue}{#1}}


\newcommand{\OmT}{\Omega\times (0,T)}
\let\pa\partial
\let\na\nabla
\newcommand{\red}[1]{\textcolor{red}{#1}}
\newcommand{\cD}{\mathcal{D}}
\let\eps\varepsilon
\newcommand{\wen}{w^{(\varepsilon,N)}}
\newcommand{\OmTc}{\overline{\Omega}\times [0,T]}
\newcommand{\rrhoA}{\sqrt{\rho_A}}
\newcommand{\rrhoB}{\sqrt{\rho_B}}
\newcommand{\rrhoAB}{\sqrt{\rho_A\rho_B}}

% Theorems
%-----------------------------------------------------------------
\newtheorem{thm}{Theorem}[section]
% \newtheorem{thm}{Theorem}
\newtheorem{cor}[thm]{Corollary}
\newtheorem{lem}[thm]{Lemma}
\newtheorem{prp}[thm]{Proposition}
\newtheorem{claim}[thm]{Claim}
% \newtheorem{hyp}[thm]{Hypothesis}
\newtheorem{hyp}{Hypothesis}
\theoremstyle{definition}
\newtheorem{dfn}[thm]{Definition}
\theoremstyle{remark}
\newtheorem{remark}[thm]{Remark}
\newtheorem{ex}[thm]{Example}
\newtheorem{q}{Question}
\newenvironment{ans}{\paragraph{Answer:}}{\hfill$\square$}

\hypersetup{pdftitle={Measure Theory}}
\hypersetup{pdfauthor={Josephine Evans }}


\author{
Josephine Evans
}
\title{Measure Theory: Exercises (not for credit)-with solutions}
%\date{}

\makeindex

\begin{document}
\maketitle

\begin{q}
Find the $\sigma$-algebra on $\mathbb{R}$ which is generated by the collection of all one-point sets.
\end{q}
\begin{ans}
This is the collection of all sets with a countable number of elements and their complements. To prove this we can see that we can create any set with a countable number of elements as the countable union of the one point sets of all its elements and the complements can then be found by taking complements. We also need to show that the $\sigma$-algebra generated by the one-point sets doesn't contain any other sets. As the intersection of $\sigma$-algebras is a $\sigma$-algebra it is sufficient to show that the collection of all countable sets and their complements is indeed a $\sigma$-algebra. As we have seen we can do this by showing that it is closed under countable unions and complements. The complement part is immediately clear. The countable union part is one of those things where its easier to just think about it a bit than write something down. I'm happy for people just to claim its clear!
\end{ans}

\vspace{10pt}

\begin{q}
Find an example to show that the union of a collection of $\sigma$-algebras is not necessarilly a $\sigma$-algebra.
\end{q}
\begin{ans}
Take the $\sigma$ algebras on $E$ $(\emptyset, A, A^c, E)$ and $(\emptyset, B, B^c, E)$ where $A \neq B$ and $A \neq B^c$. These are both $\sigma$-algebras but their union is not as it doesn't contain $A \cup B$ (and other things!).
\end{ans}

\vspace{10pt}

\begin{q}
Prove that if $\mathcal{E}$ is both a $d$-system and $\pi$-system then it is a $\sigma$-algebra. Use this to prove Dynkin's $\pi$-system lemma that if $\mathcal{A}$ is a $\pi$-system then any $d$-system containing $\mathcal{A}$ also contains $\sigma(A)$. \emph{Hint: Consider $\mathcal{D}$ the intersection of all $d$-systems containing $\mathcal{A}$, and $\mathcal{D}' = \{B \in \mathcal{D}\,:\, B \cap A \in \mathcal{D} \, \forall \, A \in \mathcal{A}\}$ and $\mathcal{D}'' = \{ B \in \mathcal{D} \, :\, B \cap A \in \mathcal{D} \, \forall A \in \mathcal{D}\}$ and show that they are all $d$-systems.}
\end{q}
\begin{ans} This one is hard! First we argue that anything that is both a $\pi$-system and $d$-system must be a $\sigma$-algebra. By definition it will contain both $E$ and $\emptyset$. Since it is a $d$-system if $A \in \mathcal{E}$ then $A^c = E \setminus A \in \mathcal{E}$. Now we need to show that it countains countable unions. To do this we need to first show that it countains finite unions, as $A \cup B = (A^c \cap B^c)^c$ and $\mathcal{E}$ is closed under complements and also intersections (as it is a $\pi$-system) then it is closed under finite unions. Now find some sequence $(A_n)_n$, now since we are closed under finite unions $B_n = \bigcup_{k=1}^n A_k \in \mathcal{E}$ and the $B_n$ are an increasing sequence with $\bigcup_n B_n = \bigcup_n A_n$. Now since $\mathcal{E}$ is a $d$-system we have $\bigcup_n B_n \in \mathcal{E}$. This concludes the proof.

Let $\mathcal{D}$ be the intersection of all $d$-systems that contains $\mathcal{A}$. We can see that $\mathcal{D}$ is itself a $d$-system (its exactly the same as the argument for $\sigma$-algebras). We will show that $\mathcal{D}$ is in fact a $\pi$-system. In order to do this let us look at $\mathcal{D}'$ mentioned in the hint. We can see that $\mathcal{A} \subseteq \mathcal{D}'$ as $\mathcal{A}$ is a $\pi$-system. Let us show that $\mathcal{D}'$ is a $d$-system. $E \in \mathcal{D}'$. Suppose that $B_1, B_2 \in \mathcal{D}'$  with $B_2 \subset B_1$, then we have
\[ (B_1 \setminus B_1) \cap A = (B_1 \cap A) \setminus (B_2 \cap A). \] Since $B_1, B_2$ are in $\mathcal{D}'$ then $B_1 \cap A, B_2 \cap A \in \mathcal{D}$ and as $\mathcal{D}$ is a $d$-system this means $(B_1 \cap A) \setminus (B_2 \cap A) \in \mathcal{D}$ therefore $B_1 \setminus B_2 \in \mathcal{D}'$. Now suppose $B_n \uparrow B$ and $B_n \in \mathcal{D}'$ for every $n$ then
\[ B\cap A = \bigcup_n (B_n \cap A), \] and $B_n \cap A \uparrow B cap A$, since $B_n \in \mathcal{D}'$ we have $B_n \cap A \in \mathcal{D}$ and since $\mathcal{D}$ is a $d$-system this means $B \cap A \in \mathcal{D}$ and consequently $B \in \mathcal{D}'$. This concludes the proof that $\mathcal{D}'$ is a $d$-system. As $\mathcal{D}$ was the smallest $d$-system containing $\mathcal{A}$ and $\mathcal{A} \subseteq \mathcal{D}' \subseteq \mathcal{D}$ this shows that $\mathcal{D} = \mathcal{D}'$.

Now we would like to show that $\mathcal{D}''$ is a $d$-system. We do this in essentially exactly the same way as we did for $\mathcal{D}'$. This shows that $\mathcal{D}''= \mathcal{D}$ which shows that $\mathcal{D}$ is a $\pi$-system. 
\end{ans}

\vspace{10pt}

\begin{q}
Suppose that $(E, \mathcal{E}, \mu)$ is a measure space. Prove the inclusion-exclusion formula
\[ \mu(A_1 \cup A_2 \cup \dots A_n) = \sum_{k=1}^n \mu(A_k)- \sum_{k \neq j}^n \mu(A_k \cap A_j) + \dots + (-1)^n \mu(A_1 \cap A_2 \cap \dots A_n). \]
\end{q}
\begin{ans}
We need to use finite additivity. In the case $n=2$ we have $A_1 \cup A_2 = (A_1 \setminus (A_1 \cap A_2))\cup(A_2 \setminus (A_1 \cap A_2)) \cup (A_1 \cap A_2)$ where all these intersectons are disjoint. Therefore we have
\[ \mu(A_1 \cup A_2) = \mu(A_1 \setminus (A_1 \cap A_2)) + \mu(A_2 \setminus (A_1 \cap A_2)) + \mu(A_1 \cap A_2). \] Also by finite additivity we know that $\mu(A_1) = \mu(A_1 \setminus (A_1 \cap A_2)) + \mu(A_1 \cap A_2)$ rearranging this gives $\mu(A_1 \setminus (A_1 \cap A_2)) = \mu(A_1) - \mu(A_1 \cap A_2)$. We can put this all together to get the result for $n=2$. The result for larger $n$ is then proved by induction in exactly the same way as the inclusion exclusion formula you have seen before.
\end{ans}

\vspace{10pt}


\begin{q}
Let $\mu$ be the measure on $\mathbb{R}$ defined by setting $\mu(A)$ to be the number of rationals in the set $A$ (where $\mu(A) = \infty$ if there are infinitely many rationals). Show that $\mu$ is a $\sigma$-finite measure which gives every open interval infinite measure. 
\end{q}
\begin{ans}
Showing this is a measure is a straightforward checking exercise (Unless I've screwed up!). The hard part is showing that $\mu$ is $\sigma$-finite. In order to do this we take a countable partition of the reals and pair them up with the rationals. Let $A_n = ([n,n+1) \cup [-n-1, -n))$ and let $B_n = A_n \setminus (A_n \cap \mathbb{Q})$. Since $\mathbb{Q}$ is countable we can enumerate the rationals $q_1, q_2, \dots$ then let $E_n = B_n \cup \{q_n\}$. Then we can see that $\bigcup_n E_n = \bigcup_n B_n \cup \mathbb{Q} = \bigcup_n A_n = \mathbb{R}$ and $\mu(E_n) = 1$ for every $n$. So this shows it is $\sigma$-finite.
\end{ans}


\vspace{10pt}

\begin{q}
Let $\mu$ be a finitely additive set function on a $\sigma$-algebra, $\mathcal{E}$ in a set $E$ with $\mu(E) < \infty$. Show that $\mu$ is countably additive \emph{if and only if} for any decreasing sequence of sets $A_n$ with $\bigcap_n A_n = \emptyset$ and $\mu(A_1) < \infty$ then we have $\mu(A_n) \rightarrow 0$.
\end{q}

\begin{ans}
Let $B_n = A_n \setminus A_{n+1}$ then the $B_n$ are disjoint with $\bigcup_n B_n = A_1$. Therefore if $\mu$ is countably additive we have $\sum_n \mu(B_n) = \mu(A_1)<\infty$ therefore $\sum_{n \geq N} \mu(B_n) \rightarrow 0$ as $N \rightarrow \infty$ and $\sum_{n \geq N} \mu(B_n) = \mu(A_N)$.

In the other direction, let $B_n$ be a sequence of disjoint sets and let $B = \bigcup_n B_n$ and $A_n = B \setminus \bigcup_{k=1}^n B_n$ then $A_n$ is a sequence of sets which decends to $0$, hence $\mu(A_n) \rightarrow 0$. We have that $B = \bigcup_{k=1}^n B_k \cup A_n$ so $\mu(B) = \sum_{k=1}^n \mu(B_k) + \mu(A_n)$ taking limits as $n \rightarrow 0$ and using the fact that $\mu(A_n) \rightarrow 0$ gives the result. 
\end{ans}

\vspace{10pt}

\begin{q}
Let $(E, \mathcal{E}, \mu)$ be a measure space. We call a set $N$ a \emph{null set} if there exists $B \in \mathcal{E}$ with $N \subseteq B$ and $\mu(B) = 0$. We write $\mathcal{N}$ for the collection of all subsets. Define the collection
\[ \mathcal{E}^\mu = \{ A \cup N \, :\, A \in \mathcal{E}, N \in \mathcal{N}\}. \] Show that $\mathcal{E}^\mu$ is a $\sigma$-algebra and the extension of $\mu$ to $\mathcal{E}^\mu$ defined by $\mu(A \cup N) = \mu(A)$ is a measure. We call $\mathcal{E}^\mu$ the \emph{completion of $\mathcal{E}$ with respect to $\mu$}.
\end{q}

\begin{ans}
We need to show that $\mathcal{E}^\mu$ is closed under complements and countable unions. $(A \cup N)^c = A^c \cap N^c$ and there is some set $B \in \mathcal{E}$ with $N \subseteq B$ and $\mu(B) =0$, so $N^c = B^c \cup (B \setminus N)$. We have that $(A \cup N)^c = (A^c \cap B^c) \cup (A^c \cap (B \setminus N))$ and $A^c \cap (B \setminus N) \subseteq B$ so $A^c \cap (B \setminus N) \in \mathcal{N}$. Now suppose that $(A_n \cup N_n)_n$ is a sequence in $\mathcal{E}^\mu$ then $\bigcup_n (A_n \cup N_n)  = \bigcup_n (A_n) \cup \bigcup_n N_n$. For every $n$ there is a $B_n \in \mathcal{E}$ with $\mu(B_n) = 0$ therefore $\bigcup_n N_n \subseteq \bigcup_n B_n$ and by countable additivity $\bigcup_n B_n \in \mathcal{E}$ with $\mu(\bigcup_n B_n) = 0$ so $\bigcup_n N_n$ is a null set.

Now we want to show that the extension of $\mu$ to $\mathcal{E}^\mu$ is a measure. Therefore we need to show that it is countably additive. Let $(A_n \cup N_n)$ be a sequence of disjoint sets then 
\[ \mu( \bigcup_n(A_n \cup N_n)) = \mu(\bigcup_n A_n) = \sum_n \mu(A_n) = \sum_n \mu(A_n \cup N_n). \] We also need to show that it is well defined. Suppose that $A_1 \cup N_1 = A_2 \cup N_2$ where $A_1, A_2 \in \mathcal{E}$ and $N_1, N_2 \in \mathcal{N}$. Then $A_1 \cup N_1 = (A_1 \cap A_2) \cup N_1 \cup N_2$, and we woud like to show that $\mu(A_1) = \mu(A_1 \cap A_2)$. We have that $A_1 \setminus (A_1 \cap A_2) \subset N_1 \cup N_2$. Therefore there are sets $B_1, B_2 \in \mathcal{E}$ with $\mu(B_1) = \mu(B_2) = 0$ and $N_1 \subseteq B_1, N_2 \subseteq B_2$. This implies that $A_1 \setminus (A_1 \cap A_2)  \subseteq B_1 \cup B_2$ and by monotonicity of measures $\mu(A_1 \setminus (A_1 \cap A_2)) = 0$ therefore $\mu(A_1) = \mu(A_1 \cap A_2)$.
\end{ans}
\end{document}