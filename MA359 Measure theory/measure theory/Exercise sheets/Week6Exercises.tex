\documentclass[11pt]{article}

\usepackage{mathtools}
\usepackage{amsmath, amsthm, amsfonts,amssymb}
\usepackage{enumitem}
\usepackage{graphicx}
\usepackage{colortbl}
\usepackage{tikz}
\usepackage[utf8]{inputenc}
\usepackage{esint}
\usepackage{mathrsfs}
\usepackage{subfig,float}
\usepackage[T1]{fontenc}
\usepackage{mathrsfs}  
\usepackage{bbm} 
\usepackage{enumitem}
\usepackage{enumerate}
\usepackage{mathtools}
\usepackage{dsfont}
\newcommand{\set}[1]{\left\{#1\right\}}

\def\grad{\nabla}
\DeclareMathOperator{\dive}{div}
\DeclareMathOperator{\supp}{supp}
\DeclareMathOperator{\essup}{ess\,sup}
\DeclareMathOperator{\Lip}{Lip}
\DeclareMathOperator{\sgn}{sgn}

% Notation for differentials
\def\d{\,\mathrm{d}}
\def\dv{\d v}
\def \ddt{\frac{\mathrm{d}}{\mathrm{d}t}}
\def \ddt{\frac{\mathrm{d}}{\mathrm{d}t}}
\def \ddr{\frac{\mathrm{d}}{\mathrm{d}r}}
\newcommand{\sign}{\text{sign}}
\DeclareMathOperator{\hess}{Hess}

% Generate a PDF with hyperlinks in references.
\usepackage[colorlinks=true,linkcolor=blue,citecolor=blue,urlcolor=blue,breaklinks]{hyperref}

% Bibliography
%-----------------------------------------------------------------

% This uses a bibliography style which hyperlinks the paper titles to
% the paper URL specified in the bibtex file. It also uses natbib,
% which cites papers by name such as Euler (1770) instead of [17].
\usepackage{hyperref}
\usepackage{breakurl}
\usepackage[square,sort,comma,numbers]{natbib}
%\usepackage{natbib}
\usepackage{url}
%\bibliographystyle{plainnat-linked}
\bibliographystyle{plain}

%\usepackage[notcite,notref]{showkeys}
%\usepackage{hyperref}
%\usepackage{breakurl}
\usepackage[square,sort,comma,numbers]{natbib}
%\usepackage{natbib}
%\usepackage{url}
%\usepackage[colorlinks=blue,linkcolor=blue,citecolor=blue,urlcolor=blue,breaklinks]{hyperref}
\bibliographystyle{plain}
\DeclarePairedDelimiter\abs{\lvert}{\rvert}%
\DeclarePairedDelimiter\norm{\lVert}{\rVert}%


\addtolength{\oddsidemargin}{-.875in}
\addtolength{\evensidemargin}{-.875in}
\addtolength{\textwidth}{1.75in}

\addtolength{\topmargin}{-.875in}
\addtolength{\textheight}{1.75in}

% Shortcuts
%-----------------------------------------------------------------

%\newcommand{\abs}[1]{\left\mid#1\right\mid}
%\newcommand{\ap}[1]{\left\langle#1\right\rangle}
%\newcommand{\norm}[1]{\left\mid#1\right\mid}
% \newcommand{\tnorm}[1]{\left\mid\!\left\mid\!\left\mid#1\right\mid\!\right\mid\!\right\mid}

\definecolor{lpink}{rgb}{0.96, 0.76, 0.76}
\definecolor{dpink}{rgb}{0.97, 0.51, 0.47}
\definecolor{sky}{rgb}{0.53, 0.81, 0.92}
\definecolor{salmon}{rgb}{1.0, 0.55, 0.41}
\definecolor{orman}{rgb}{0.24, 0.7, 0.44}
\definecolor{aciksari}{rgb}{0.91, 0.84, 0.42}
\definecolor{dgrey}{rgb}{0.52, 0.52, 0.51}

\def\R{\mathbb{R}}
\def\C{\mathbb{C}}
\def\P{\mathscr{P}}
\def\NN{\mathbb{N}}
\def\Q{\mathbb{Q}}

\def\ird{\int_{\mathbb{R}^N}}
\def\d{\,\mathrm{d}}
\def\dx{\,\mathrm{d}x}
\def\dy{\,\mathrm{d}y}
\def\p{\,\partial}
\newcommand{\en}{\mathcal{H}}
\newcommand{\havva}[1]{{\textcolor{blue}{[\textbf{H:} #1]}}}

% Operators

\def\grad{\nabla}
\def\weakto{\rightharpoonup}

\DeclareMathOperator{\divergence}{div}
%\newcommand{\dv}[1]{\divergence \left(#1\right)}
%
%\DeclareMathOperator{\supp}{supp}
%\DeclareMathOperator{\essup}{ess\,sup}
%\DeclareMathOperator{\Lip}{Lip}
\DeclareMathOperator{\law}{law}

\DeclareMathOperator{\Pf}{Pf.}
\DeclareMathOperator{\Fp}{Fp.}
\DeclareMathOperator{\pv}{pv.}

\newcommand{\for}{\quad \text{ for all }}
\newcommand{\tb}[1]{\textcolor{blue}{#1}}


\newcommand{\OmT}{\Omega\times (0,T)}
\let\pa\partial
\let\na\nabla
\newcommand{\red}[1]{\textcolor{red}{#1}}
\newcommand{\cD}{\mathcal{D}}
\let\eps\varepsilon
\newcommand{\wen}{w^{(\varepsilon,N)}}
\newcommand{\OmTc}{\overline{\Omega}\times [0,T]}
\newcommand{\rrhoA}{\sqrt{\rho_A}}
\newcommand{\rrhoB}{\sqrt{\rho_B}}
\newcommand{\rrhoAB}{\sqrt{\rho_A\rho_B}}

% Theorems
%-----------------------------------------------------------------
\newtheorem{thm}{Theorem}[section]
% \newtheorem{thm}{Theorem}
\newtheorem{cor}[thm]{Corollary}
\newtheorem{lem}[thm]{Lemma}
\newtheorem{prp}[thm]{Proposition}
\newtheorem{claim}[thm]{Claim}
% \newtheorem{hyp}[thm]{Hypothesis}
\newtheorem{hyp}{Hypothesis}
\theoremstyle{definition}
\newtheorem{dfn}[thm]{Definition}
\theoremstyle{remark}
\newtheorem{remark}[thm]{Remark}
\newtheorem{ex}[thm]{Example}
\newtheorem{q}{Question}
\newenvironment{ans}{\paragraph{Answer:}}{\hfill$\square$\vspace{10pt}}

\hypersetup{pdftitle={Measure Theory}}
\hypersetup{pdfauthor={Josephine Evans }}


\author{
Josephine Evans
}
\title{Measure Theory: Exercises (not for credit)}
%\date{}

\makeindex

\begin{document}
\maketitle
%\begin{q}
%Let $X_n$ be a sequence of random variables taking values in $\mathbb{R}$ defined on the same probability space $(\Omega, \mathcal{F}, \mathbb{P})$.. Define $F_{X_n}(x) = \mathbb{P}(X_n \in (-\infty, x])$. We are going to show that $X_n \rightarrow X$ in measure (that is to say for any $\epsilon >0, \mathbb{P}(|X_n -X|>\epsilon) \rightarrow 0$ then $F_{X_n}(x) \rightarrow F_X(x)$ at every $x$ where $F_X(x)$ is continuous. We can summarise this as saying that for real valued random variables convergence in measure imples convergence in distribution.
%\begin{itemize}
%\item Show that for each $x$ where $F_X(x)$ is continuous, and each $\epsilon$ there exists a $\delta$ such that $F_X(x-\delta) \geq F(x)-\epsilon$ and $F_X(x+\delta) \leq F_X(x) + \epsilon/2$.
%\item Now explain why for every $x$ $F_{X_n}(x) \leq F_X(x+\delta) + \mathbb{P}(|X_n - X| > \delta)$ and $F_{X_n}(x) \geq F_X(x-\delta) - \mathbb{P}(|X_n -X| > \delta)$
%\item Use the two parts above to show that $X_n \rightarrow X$ in measure implies $F_{X_n}(x) \rightarrow F_{X}(x)$ whenever $F_X(x)$ is continuous.
%\end{itemize}
%\end{q}

\begin{q}[The devil's staircase]
In this question we construct a function which is continuous, flat almost everywhere and increases from 0 to 1 as $x$ goes from 0 to 1 (This is quite a lot like doing research, you are only making progress at a measure 0 amount of time!). First we construct the Cantor set recursively. Let $C_0 = [0,1], C_1 = [0,1/3] \cup [2/3,1], C_2 = [0,1/9]\cup[2/9,1/3] \cup [2/3,7/9]\cup[8/9,1], \dots$ where each $C_n$ is constructed from $C_{n-1}$ by removing the middle thirds of each of the closed intervals making up $C_{n-1}$. Let us write $C= \bigcap_n C_n$, so $C_n \downarrow C$ that is to say $C_1 \supset C_2 \supset \dots$ and $C = \bigcap_n C_n$. 
\begin{itemize}
\item Show that $C$ is uncountable.
\item Show that $\lambda(C) =0$.
\item Define $F_n(x) = \lambda(C_n \cap [0,x])/\lambda(C_n)$, show that $F(x) = \lim_n F_n(x)$ exists. \emph{hint: try and find a reccurrence relationship for $F_n$ in terms of $F_{n-1}$ then use this to show $F_n$ is a Cauchy sequence with the uniform norm on functions}
\item Show that the function $F$ is continuous for all $x$ with $F(0) = 0$ and $F(1) = 1$.
\item Show that for lebesgue almost every $x \in [0,1]$ we have that $F(x)$ is differentiable with $F'(x) = 0$.
\end{itemize}
\end{q}

%\begin{ans}
%First let us show that the Cantor set is uncountable. We can write every element of $[0,1]$ as an expansion in base $3$. That is to say $x= k_1/3 + k_2/9 + k_3/27 + \dots + k_n/3^n + \dots$. Then $x \in C$ if and only if $k_n \in \{0,2\}$ for every $n$. Therefore the cardinality of the Cantor set is the same as the cardinality of all sequences or numbers which are either $0$ or $2$, which is uncountable by the standard Cantor diagonal argument to show uncountability of the reals.
%
%Now let us show that $\lambda(C) = 0$. We have that $\lambda(C_0) = 1$ so we can apply our continuity of measure theorem to get that $\lambda(C) = \lim_n \lambda(C_n)$. Then we claim that $\lambda(C_n) = (2/3) \lambda(C_{n-1})$ since for each interval making up $C_{n-1}$ we have removed a third of it. Working itteratively this gives that $\lambda(C_n) = (2/3)^n \lambda(C_0) = (2/3)^n$ therefore $\lambda(C) = \lim_n (2/3)^n = 0$. Notice here that we have shown the existence of a measure 0 set of uncountable cardinality.
%
%This step is really tricky. We can define find a reccurence relationship for $F_n(x)$ using the fractal-like property of the Cantor set
%\[  F_n(x) = \left\{ \begin{array}{ll} F_{n-1}(3x)/2 & x \in [0,1/3] \\
%1/2 & x \in [1/3,2/3] \\
%1/2(1+F_{n-1}(3x-2)) & x \in [2/3,1] \end{array}\right. \] Now we want to look at $F_{n+1}(x) - F_n(x)$ we can split into the same 3 cases and get
%\[ F_{n+1}(x) - F_n(x) = \left\{ \begin{array}{ll} (F_n(3x)-F_{n-1}(3x))/2 & x \in [0,1/3] \\
%1/2 -1/2 & x \in [1/3,2/3] \\
%(F_n(3x-2)+1 -(F_{n-1}(3x-2) +1))/2 & x \in [2/3,1] \end{array}\right. \] Then we have that $|F_n(x) -F_{n-1}(x)| \leq 2^{-n+1}|F_1(x) -F_0(x)| \leq 2^{-n+1}$ so $F_n(x)$ is a uniformly Cauchy sequence.
%
%The next step relies on the previous one. We can check that $F_n(x)$ is continuous for each $x$. Then we know that the uniform limit of continuous functions is also continuous. $F_n(0) = \lambda(\emptyset)/\lambda(C_n) = 0$ so $F(0) = 0$ and $F_n(1) = \lambda(C_n)/\lambda(C_n) = 1$ so $F(1) = 1$.
%
%Now suppose that $x \notin C$ then for $n$ sufficiently large $x \notin C_n$ the interval $[0,1] \setminus C_n$ is open so there exists a $\delta>0$ such that $(x-\delta, x+\delta) \subset [0,1] \setminus C_n$. Then $F_m$ will be constant on this set for any $m \geq n$ and the same value for each $m$. So $F(x)$ is constant on $(x-\delta, x+\delta)$ so $F'(x)$ exists and is $0$.
%\end{ans}






%This weeks questions sheet is a bit unusual. We're going to work through some of what is involved in proving the surprising result in the Banach-Tarski paradox. Please feel free to skip it. The results here are not examinable but it might be helpful to polish your measure theory skills. It could take a long time to write full proofs to all the questions so try and focus on capturing the essential proofs in your answer. All the answers can be found in appendix $G$ of Cohn's book.
%
%First we recall some defintions and notation. Let $G$ be a group. We write the multiplication of two elements in $G$ by $g_1g_2$ and the group property guarantees that $g_1g_2 \in G$. Let us call the identity element of $G$ $e$, and we have $ge = eg = g$. We say that a group $G$ acts on a set $X$ if we have a map from $G \times X \rightarrow X$ which we write by $(g,x) \mapsto g\cdot x$ and which satisfies $g_1\cdot(g_2 \cdot x) = (g_1 g-2) \cdot x$ and $e\cdot x = x$. Here we are really thinking about $G$ being a group of maps which encode shifts, rotations and reflections in $\mathbb{R}^d$ and $X$ being $\mathbb{R}^d$.
%
%\begin{dfn}
%Let $A$ and $B$ be two subsets of a set $X$ and let $G$ act on $X$ then we say $A$ and $B$ are equidecomposable if there exists an $n \in \mathbb{N}$ and finite disjoin unions of subsets $A_1, \dots A_n$ and $B_1, \dots, B_n$ and elements of $G$, $g_1, \dots, g_n$ such that $A= A_1 \cup A_2 \cup \dots \cup A_n$ and $B= B_1 \cup B_2 \cup \dots \cup B_n$ and $B_k = g_k A_k$ for each $k$. Here we define $g_kA_k = \{ g_kx \,:\, x \in A_k\}$. We can write a map $f$ which defines this equidecomposability condition by setting $f(x) = g_k x $ whenever $x \in A_k$.
%\end{dfn}
%
%\begin{q}
%Show that equidecomposability defines an equivalence relationship on the subsets of $X$.
%\end{q}
%
%\begin{q}
%Let $A$ and $B$ be two subsets of $X$, show that if $A$ is equidecomposable with a subset of $B$ and $B$ is equidecomposable with a subset of $A$ then $A$ and $B$ are equidecomposable. \emph{Hint: define $a \in A$ to be a parent of $b \in B$ if $b = f(a)$ for $f$ a map defining the equidecomposablility condition. Then look at parents, grandparents, great-great grandparents of each element then split in the cases where $a$ has an odd number of ancestors, an even number of ancestors or infinitely many ancestors then split $A$ into these sets to form a new map for equidecomposability between the whole of $A$ and $B$.}
%\end{q}
%
%We can now write down the Banach-Tarski paradox using this notion.
%\begin{thm}[Banach-Tarski Theorem]
%Let $A$ and $B$ be subsets of $\mathbb{R}^3$ which are bounded with non-empty interiors and let $G$ be the group of rigid motions (shifts, rotations and reflections) show that $A$ and $B$ are $G$-equidecomposable. 
%\end{thm}
%\begin{q}
%Explain why the theorem above means that we can decompose a pea into finitely many sets and rigidly move them around to form the sun.
%\end{q}
%\begin{dfn}
%Suppose a group $G$ acts on a set $X$. We say that a subset $A$ or $X$ is $G$-paradoxical if there exists $A_1$ and $A_2$ disjoint with $A_1 \cup A_2 = A$ such that $A_1$ is equidecomposable with $A$ and $A_2$ is also equidecomposable with $A$.
%\end{dfn}
%\begin{q}
%Using the previous questions show that if $A \subseteq X$ has two subsets $B_1$ and $B_2$ which are disjoint and such that $B_1$ is equidecomposable with $A$ and $B_2$ is equidecomposable with $A$ then $A$ is paradoxical.
%\end{q}
%
%\begin{q}
%Show that the conclusion of the Banach-Tarski theorem holds if and only if the ball $\{x \,:\, |x| \leq 1\}$ is $G$-paradoxical, where $G$ is the group of rigid motions.
%\end{q}
\end{document}