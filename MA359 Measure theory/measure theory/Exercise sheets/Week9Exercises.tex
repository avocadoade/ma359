\documentclass[11pt]{article}

\usepackage{mathtools}
\usepackage{amsmath, amsthm, amsfonts,amssymb}
\usepackage{enumitem}
\usepackage{graphicx}
\usepackage{colortbl}
\usepackage{tikz}
\usepackage[utf8]{inputenc}
\usepackage{esint}
\usepackage{mathrsfs}
\usepackage{subfig,float}
\usepackage[T1]{fontenc}
\usepackage{mathrsfs}  
\usepackage{bbm} 
\usepackage{enumitem}
\usepackage{enumerate}
\usepackage{mathtools}
\usepackage{dsfont}
\newcommand{\set}[1]{\left\{#1\right\}}

\def\grad{\nabla}
\DeclareMathOperator{\dive}{div}
\DeclareMathOperator{\supp}{supp}
\DeclareMathOperator{\essup}{ess\,sup}
\DeclareMathOperator{\Lip}{Lip}
\DeclareMathOperator{\sgn}{sgn}

% Notation for differentials
\def\d{\,\mathrm{d}}
\def\dv{\d v}
\def \ddt{\frac{\mathrm{d}}{\mathrm{d}t}}
\def \ddt{\frac{\mathrm{d}}{\mathrm{d}t}}
\def \ddr{\frac{\mathrm{d}}{\mathrm{d}r}}
\newcommand{\sign}{\text{sign}}
\DeclareMathOperator{\hess}{Hess}

% Generate a PDF with hyperlinks in references.
\usepackage[colorlinks=true,linkcolor=blue,citecolor=blue,urlcolor=blue,breaklinks]{hyperref}

% Bibliography
%-----------------------------------------------------------------

% This uses a bibliography style which hyperlinks the paper titles to
% the paper URL specified in the bibtex file. It also uses natbib,
% which cites papers by name such as Euler (1770) instead of [17].
\usepackage{hyperref}
\usepackage{breakurl}
\usepackage[square,sort,comma,numbers]{natbib}
%\usepackage{natbib}
\usepackage{url}
%\bibliographystyle{plainnat-linked}
\bibliographystyle{plain}

%\usepackage[notcite,notref]{showkeys}
%\usepackage{hyperref}
%\usepackage{breakurl}
\usepackage[square,sort,comma,numbers]{natbib}
%\usepackage{natbib}
%\usepackage{url}
%\usepackage[colorlinks=blue,linkcolor=blue,citecolor=blue,urlcolor=blue,breaklinks]{hyperref}
\bibliographystyle{plain}
\DeclarePairedDelimiter\abs{\lvert}{\rvert}%
\DeclarePairedDelimiter\norm{\lVert}{\rVert}%


\addtolength{\oddsidemargin}{-.875in}
\addtolength{\evensidemargin}{-.875in}
\addtolength{\textwidth}{1.75in}

\addtolength{\topmargin}{-.875in}
\addtolength{\textheight}{1.75in}

% Shortcuts
%-----------------------------------------------------------------

%\newcommand{\abs}[1]{\left\mid#1\right\mid}
%\newcommand{\ap}[1]{\left\langle#1\right\rangle}
%\newcommand{\norm}[1]{\left\mid#1\right\mid}
% \newcommand{\tnorm}[1]{\left\mid\!\left\mid\!\left\mid#1\right\mid\!\right\mid\!\right\mid}

\definecolor{lpink}{rgb}{0.96, 0.76, 0.76}
\definecolor{dpink}{rgb}{0.97, 0.51, 0.47}
\definecolor{sky}{rgb}{0.53, 0.81, 0.92}
\definecolor{salmon}{rgb}{1.0, 0.55, 0.41}
\definecolor{orman}{rgb}{0.24, 0.7, 0.44}
\definecolor{aciksari}{rgb}{0.91, 0.84, 0.42}
\definecolor{dgrey}{rgb}{0.52, 0.52, 0.51}

\def\R{\mathbb{R}}
\def\C{\mathbb{C}}
\def\P{\mathscr{P}}
\def\NN{\mathbb{N}}
\def\Q{\mathbb{Q}}

\def\ird{\int_{\mathbb{R}^N}}
\def\d{\,\mathrm{d}}
\def\dx{\,\mathrm{d}x}
\def\dy{\,\mathrm{d}y}
\def\p{\,\partial}
\newcommand{\en}{\mathcal{H}}
\newcommand{\havva}[1]{{\textcolor{blue}{[\textbf{H:} #1]}}}

% Operators

\def\grad{\nabla}
\def\weakto{\rightharpoonup}

\DeclareMathOperator{\divergence}{div}
%\newcommand{\dv}[1]{\divergence \left(#1\right)}
%
%\DeclareMathOperator{\supp}{supp}
%\DeclareMathOperator{\essup}{ess\,sup}
%\DeclareMathOperator{\Lip}{Lip}
\DeclareMathOperator{\law}{law}

\DeclareMathOperator{\Pf}{Pf.}
\DeclareMathOperator{\Fp}{Fp.}
\DeclareMathOperator{\pv}{pv.}

\newcommand{\for}{\quad \text{ for all }}
\newcommand{\tb}[1]{\textcolor{blue}{#1}}


\newcommand{\OmT}{\Omega\times (0,T)}
\let\pa\partial
\let\na\nabla
\newcommand{\red}[1]{\textcolor{red}{#1}}
\newcommand{\cD}{\mathcal{D}}
\let\eps\varepsilon
\newcommand{\wen}{w^{(\varepsilon,N)}}
\newcommand{\OmTc}{\overline{\Omega}\times [0,T]}
\newcommand{\rrhoA}{\sqrt{\rho_A}}
\newcommand{\rrhoB}{\sqrt{\rho_B}}
\newcommand{\rrhoAB}{\sqrt{\rho_A\rho_B}}

% Theorems
%-----------------------------------------------------------------
\newtheorem{thm}{Theorem}[section]
% \newtheorem{thm}{Theorem}
\newtheorem{cor}[thm]{Corollary}
\newtheorem{lem}[thm]{Lemma}
\newtheorem{prp}[thm]{Proposition}
\newtheorem{claim}[thm]{Claim}
% \newtheorem{hyp}[thm]{Hypothesis}
\newtheorem{hyp}{Hypothesis}
\theoremstyle{definition}
\newtheorem{dfn}[thm]{Definition}
\theoremstyle{remark}
\newtheorem{remark}[thm]{Remark}
\newtheorem{ex}[thm]{Example}
\newtheorem{q}{Question}
\newenvironment{ans}{\paragraph{Answer:}}{\hfill$\square$\vspace{10pt}}

\hypersetup{pdftitle={Measure Theory}}
\hypersetup{pdfauthor={Josephine Evans }}


\author{
Josephine Evans
}
\title{Measure Theory: Exercises (not for credit)}
%\date{}

\makeindex

\begin{document}
\maketitle

\begin{q}
Suppose that $(E, \mathcal{E})$ and $(F, \mathcal{F})$ are measurable spaces. Show that the set $\mathcal{A} \subseteq \mathcal{E} \times \mathcal{F}$ with $\mathcal{A} = \{ A \times B \,:\, A \in \mathcal{E}, B \in \mathcal{F}\}$ is a $\pi$-system. 
\end{q}
%\begin{ans}
%We know that $\emptyset = \emptyset \times \emptyset$ so $\emptyset \in \mathcal{A}$. If we suppose that $C_1 = A_1 \times B_1$ and $C_2 = A_2 \times B_2$ then $C_1 \cap C_2 = (A_1 \cap A_2) \times (B_1 \cap B_2)$ (we can check this more precisesly and you probably should in an exam). Therefore $\mathcal{A}$ is a $\pi$-system.
%\end{ans}

\begin{q}
Suppose that $(E, \mathcal{E})$ and $(F, \mathcal{F})$ are measurable spaces. Let $\mathcal{A}_1 \subseteq \mathcal{E}$ and $\mathcal{A}_2 \subseteq \mathcal{F}$ be such that $\sigma(\mathcal{A}_1) = \mathcal{E}$ an $\sigma(\mathcal{A}_2) = \mathcal{F}$. Show that $\mathcal{E}\times \mathcal{F} = \sigma(\mathcal{A}_1 \times \mathcal{A}_2)$.
\end{q}
%\begin{ans}
%Let us define $\mathcal{C} = \{ A \times B \,:\, A \in \mathcal{A}_1, B \in \mathcal{F}\}$, then let us first show that $\mathcal{C} \in \sigma(\mathcal{A}_1 \times \mathcal{A}_2)$. Let us fix $A \in \mathcal{A}_1$ and define $\tilde{C}_A = \{ B \,:\, A \times B \in \sigma(\mathcal{A}_1 \times \mathcal{A}_2)$ then we know that $\mathcal{A}_2 \in \tilde{C}_A$ and we can show that $\tilde{C}_A$ is a $\sigma$-algebra as it inherits the properties of $\sigma(\mathcal{A}_1 \times \mathcal{A}_2)$. Therefore $\tilde{C}_A \supseteq \sigma(\mathcal{A}_2)$ so $\tilde{C}_A = \mathcal{F}$. Therefore $\mathcal{C} \subseteq   \sigma(\mathcal{A}_1 \times \mathcal{A}_2)$. Now fix $B \in \mathcal{F}$ and define $\bar{C}_B = \{ A \,:\, A \times B \in  \sigma(\mathcal{A}_1 \times \mathcal{A}_2)\}$. Now since we know that $\mathcal{C} \subseteq  \sigma(\mathcal{A}_1 \times \mathcal{A}_2)$ this means that $\mathcal{A}_1 \subseteq \bar{C}_B$ for each $B$. Then again $\bar{C}_B$ inherits the $\sigma$-algebra structure from  $\sigma(\mathcal{A}_1 \times \mathcal{A}_2)$ and hence is a $\sigma$-algebra so $\bar{C}_B = \mathcal{E}$. Hence $ \sigma(\mathcal{A}_1 \times \mathcal{A}_2) = \mathcal{E} \times \mathcal{F}$.
%\end{ans}

\begin{q}
Let $\mathcal{M}_1$ be the $\sigma$-algebra of Lebesgue measurable subsets of $\mathbb{R}$, and $\mathcal{M}_2$ be the $\sigma$-algebra of Lebesgue measurable subsets of $\mathbb{R}^2$. Show that $\mathcal{M}_2 \neq \mathcal{M}_1 \times \mathcal{M}_1$. 
\end{q}
%\begin{ans}
%Let $A$ be a non-lebesgue measurable subset of $\mathbb{R}$. Then let $B = \{x\} \times A$, this is contained in a line in $\mathbb{R}^2$ so is a null set therefore $B \in \mathcal{M}_2$ however $B_x = A \notin \mathcal{M}_1$. If $\mathcal{M}_2$ was equal to $\mathcal{M}_1 \times \mathcal{M}_1$ then every set $C$ in $\mathcal{M}_2$ would have $C_x \in \mathcal{M}_1$ for every $x$. Therefore $\mathcal{M}_2 \neq \mathcal{M}_1 \times \mathcal{M}_1$.
%\end{ans}

\begin{q}
Let $\mu$ be the counting measure on $(\mathbb{R}, \mathcal{B}(\mathbb{R}))$ (the measure that counts how many elements there are in a set) and let $\lambda$ be Lebesgue measure on $\mathbb{R}$. Let $f$ be the indicator function of the set $\{(x,x) \,:\, x \in \mathbb{R}\}$. Show that
\[ \int_\mathbb{R} \int_\mathbb{R} f(x,y) \mu(\mathrm{d}x) \lambda(\mathrm{d}y) \neq \int_\mathbb{R} \int_\mathbb{R}  f(x,y) \lambda(\mathrm{d}x) \mu(\mathrm{d}x). \] What part of the conditions of Fubini-Tonelli theorem doesn't hold to mean this can happen?
\end{q}
%\begin{ans}
%In the displayed equation $LHS = \infty$ and $RHS = 0$. This is possible because $(\mathbb{R}, \mathcal{B}(\mathbb{R}), \mu)$ is not $\sigma$-finite.
%\end{ans}

\begin{q}
Let $f(x,y) = 1_{x \geq 0}( 1_{y \in [x,x+1)} - 1_{y \in [x+1, x+2)})$ Show that
\[ \int_\mathbb{R} \int_\mathbb{R} f(x,y) \mathrm{d}x \mathrm{d}y \neq \int_\mathbb{R} f(x,y) \mathrm{d}y \mathrm{d}x. \] What part of the conditions of Fubini-Tonelli theorem doesn't hold to allow this to happen?
\end{q}
%\begin{ans}
%We can compute with a bit of work and get $LHS = 1$ and $RHS = 0$. This is possible because $|f(x,y)|$ is not integrable.
%\end{ans}

\begin{q}
Let $A$ be a bounded Borel subset of $\mathbb{R}$ with $\lambda(A)>0$ show that the function $x \mapsto \lambda(A \cap (x+A))$ is continuous and is non-zero on some open interval containing 0. Define $diff(A) = \{ z\,:\, z=x-y, x \in A, y \in A \}$ show that if $A$ is a Borel subset of $\mathbb{R}$ with non-zero measure then $diff(A)$ contains some open interval around 0.
\end{q}
%\begin{ans}
%Let $f(x) = 1_{A}(x), g(x) = 1_{A}(-x)$ then $f*g = \int_\mathbb{R} 1_{A}(y) 1_{A}(-x+y) \mathrm{d}y = \int_\mathbb{R} 1_{A}(y)1_{x+A}(y) \mathrm{d}y = \int_\mathbb{R} 1_{A \cap (x+A)}(y) \mathrm{d}y = \lambda(A \cap (x+A))$. Now for any $p, q$ we have that $f \in L^p, g \in L^q$ so by a result in lectures $f*g$ is continous. We also have $f*g(0) = \lambda(A) > 0$. So there must be some open interval $U$ containing $0$ such that if $x \in U$ then $\lambda(A \cap (x+A)) >0$. Then for each $x \in U$ there exists $y, z \in A$ such that $z = y+x$. Therefore $x \in diff(A)$ for every $x \in U$.
%\end{ans}


\end{document}