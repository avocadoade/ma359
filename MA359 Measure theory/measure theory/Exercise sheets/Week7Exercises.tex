\documentclass[11pt]{article}

\usepackage{mathtools}
\usepackage{amsmath, amsthm, amsfonts,amssymb}
\usepackage{enumitem}
\usepackage{graphicx}
\usepackage{colortbl}
\usepackage{tikz}
\usepackage[utf8]{inputenc}
\usepackage{esint}
\usepackage{mathrsfs}
\usepackage{subfig,float}
\usepackage[T1]{fontenc}
\usepackage{mathrsfs}  
\usepackage{bbm} 
\usepackage{enumitem}
\usepackage{enumerate}
\usepackage{mathtools}
\usepackage{dsfont}
\newcommand{\set}[1]{\left\{#1\right\}}

\def\grad{\nabla}
\DeclareMathOperator{\dive}{div}
\DeclareMathOperator{\supp}{supp}
\DeclareMathOperator{\essup}{ess\,sup}
\DeclareMathOperator{\Lip}{Lip}
\DeclareMathOperator{\sgn}{sgn}

% Notation for differentials
\def\d{\,\mathrm{d}}
\def\dv{\d v}
\def \ddt{\frac{\mathrm{d}}{\mathrm{d}t}}
\def \ddt{\frac{\mathrm{d}}{\mathrm{d}t}}
\def \ddr{\frac{\mathrm{d}}{\mathrm{d}r}}
\newcommand{\sign}{\text{sign}}
\DeclareMathOperator{\hess}{Hess}

% Generate a PDF with hyperlinks in references.
\usepackage[colorlinks=true,linkcolor=blue,citecolor=blue,urlcolor=blue,breaklinks]{hyperref}

% Bibliography
%-----------------------------------------------------------------

% This uses a bibliography style which hyperlinks the paper titles to
% the paper URL specified in the bibtex file. It also uses natbib,
% which cites papers by name such as Euler (1770) instead of [17].
\usepackage{hyperref}
\usepackage{breakurl}
\usepackage[square,sort,comma,numbers]{natbib}
%\usepackage{natbib}
\usepackage{url}
%\bibliographystyle{plainnat-linked}
\bibliographystyle{plain}

%\usepackage[notcite,notref]{showkeys}
%\usepackage{hyperref}
%\usepackage{breakurl}
\usepackage[square,sort,comma,numbers]{natbib}
%\usepackage{natbib}
%\usepackage{url}
%\usepackage[colorlinks=blue,linkcolor=blue,citecolor=blue,urlcolor=blue,breaklinks]{hyperref}
\bibliographystyle{plain}
\DeclarePairedDelimiter\abs{\lvert}{\rvert}%
\DeclarePairedDelimiter\norm{\lVert}{\rVert}%


\addtolength{\oddsidemargin}{-.875in}
\addtolength{\evensidemargin}{-.875in}
\addtolength{\textwidth}{1.75in}

\addtolength{\topmargin}{-.875in}
\addtolength{\textheight}{1.75in}

% Shortcuts
%-----------------------------------------------------------------

%\newcommand{\abs}[1]{\left\mid#1\right\mid}
%\newcommand{\ap}[1]{\left\langle#1\right\rangle}
%\newcommand{\norm}[1]{\left\mid#1\right\mid}
% \newcommand{\tnorm}[1]{\left\mid\!\left\mid\!\left\mid#1\right\mid\!\right\mid\!\right\mid}

\definecolor{lpink}{rgb}{0.96, 0.76, 0.76}
\definecolor{dpink}{rgb}{0.97, 0.51, 0.47}
\definecolor{sky}{rgb}{0.53, 0.81, 0.92}
\definecolor{salmon}{rgb}{1.0, 0.55, 0.41}
\definecolor{orman}{rgb}{0.24, 0.7, 0.44}
\definecolor{aciksari}{rgb}{0.91, 0.84, 0.42}
\definecolor{dgrey}{rgb}{0.52, 0.52, 0.51}

\def\R{\mathbb{R}}
\def\C{\mathbb{C}}
\def\P{\mathscr{P}}
\def\NN{\mathbb{N}}
\def\Q{\mathbb{Q}}

\def\ird{\int_{\mathbb{R}^N}}
\def\d{\,\mathrm{d}}
\def\dx{\,\mathrm{d}x}
\def\dy{\,\mathrm{d}y}
\def\p{\,\partial}
\newcommand{\en}{\mathcal{H}}
\newcommand{\havva}[1]{{\textcolor{blue}{[\textbf{H:} #1]}}}

% Operators

\def\grad{\nabla}
\def\weakto{\rightharpoonup}

\DeclareMathOperator{\divergence}{div}
%\newcommand{\dv}[1]{\divergence \left(#1\right)}
%
%\DeclareMathOperator{\supp}{supp}
%\DeclareMathOperator{\essup}{ess\,sup}
%\DeclareMathOperator{\Lip}{Lip}
\DeclareMathOperator{\law}{law}

\DeclareMathOperator{\Pf}{Pf.}
\DeclareMathOperator{\Fp}{Fp.}
\DeclareMathOperator{\pv}{pv.}

\newcommand{\for}{\quad \text{ for all }}
\newcommand{\tb}[1]{\textcolor{blue}{#1}}


\newcommand{\OmT}{\Omega\times (0,T)}
\let\pa\partial
\let\na\nabla
\newcommand{\red}[1]{\textcolor{red}{#1}}
\newcommand{\cD}{\mathcal{D}}
\let\eps\varepsilon
\newcommand{\wen}{w^{(\varepsilon,N)}}
\newcommand{\OmTc}{\overline{\Omega}\times [0,T]}
\newcommand{\rrhoA}{\sqrt{\rho_A}}
\newcommand{\rrhoB}{\sqrt{\rho_B}}
\newcommand{\rrhoAB}{\sqrt{\rho_A\rho_B}}

% Theorems
%-----------------------------------------------------------------
\newtheorem{thm}{Theorem}[section]
% \newtheorem{thm}{Theorem}
\newtheorem{cor}[thm]{Corollary}
\newtheorem{lem}[thm]{Lemma}
\newtheorem{prp}[thm]{Proposition}
\newtheorem{claim}[thm]{Claim}
% \newtheorem{hyp}[thm]{Hypothesis}
\newtheorem{hyp}{Hypothesis}
\theoremstyle{definition}
\newtheorem{dfn}[thm]{Definition}
\theoremstyle{remark}
\newtheorem{remark}[thm]{Remark}
\newtheorem{ex}[thm]{Example}
\newtheorem{q}{Question}
\newenvironment{ans}{\paragraph{Answer:}}{\hfill$\square$\vspace{10pt}}

\hypersetup{pdftitle={Measure Theory}}
\hypersetup{pdfauthor={Josephine Evans }}


\author{
Josephine Evans
}
\title{Measure Theory: Exercises (not for credit)}
%\date{}

\makeindex

\begin{document}
\maketitle
Questions 5 and 6 are long and not super important so only do them if you are enjoying doing exercise sheets!

\begin{q}
Suppose that $f$ is a measurable function from $\mathbb{R} \rightarrow \mathbb{R}$, show that $g$ defined by $g(x) = f(x+\tau)$ is also measurable, where $\tau$ is a fixed real number.
\end{q}


\begin{q}
Show the restriction of measure makes sense. That is to say if $(E, \mathcal{E}, \mu)$ is a measure space and $A \in \mathcal{E}$ then show that $\mathcal{E}_A = \{ B  \in \mathcal{E} \,:\, B \subset A\}$ is a $\sigma$-algebra and $\mu_A = \mu|_{\mathcal{E}_A}$ is a measure. Show further that if $g$ is a non-negative measurable function then $\mu_A(g) = \mu(g1_A)$. 
\end{q}

\begin{q}
Show that the function $sin(x)/x$ is not Lebesgue integrable over $[1,\infty]$ but the limit as $n \rightarrow \infty$ of 
\[ \int_1^n \sin(x)/x \mathrm{d}x. \]
\end{q}

\begin{q}
 Let $(E, \mathcal{E})$ and $(F, \mathcal{F})$ be measureable spaces. We call $K$ a kernel on if for each $x \in E$ the function $A \mapsto K(x, A)$ is a measure on $(F, \mathcal{F})$, and for every $A \in \mathcal{F}$ the function $x \mapsto K(x,A)$ is a measurable function on $(E, \mathcal{E})$. Suppose that $K$ is a kernel, $\mu$ is a measure on $(E, \mathcal{E})$ and $f$ is a $[0,\infty]$ valued measurable function on $(F, \mathcal{F})$. Show that
\begin{itemize}
\item $A \mapsto \int K(x,A) \mu(\mathrm{d}x)$ is a measure on $(F, \mathcal{F})$,
\item $x \mapsto \int f(y) K(x, \mathrm{d}y)$ is a measurable function on $(E, \mathcal{E})$ 
\item If $\nu$ is the measure defined by $\nu(A) = \int K(x,A) \mu(\mathrm{d}x)$ then $\nu(f) = \int \int f(y) K(x, \mathrm{d}y) \mu(\mathrm{d}x)$. 
\end{itemize}
\end{q}

\begin{q}
Recall in the last exercise sheet we constructed the devils staircase function $f: [0,1] \rightarrow [0,1]$ which is continuous and non-decreasing and flat everywhere except the Cantor set. Define a function by
\[ g(y) = \inf\{ x \in [0,1] \,:\, f(x) = y\}, \] such a function is well defined since as $f$ is continuous the intermediate value theorem shows there must be at least one $x$ such that $f(x) = y$. 
\begin{itemize}
\item Show that the range of $g$ is contained inside the Cantor set.
\item Show that $f(g(y)) = y$ and that hence $g$ is injective. 
\item Let $A$ be a non-Lebesgue measurable subset of $[0,1]$ (such as was constructed by Vitali) show that $B = g(A)$ is Lebesgue measurable.
\item Show that $B$ is not Borel measurable. 
\item Show that $h(x) = 1_B(x)$ is Riemann integrable (note this shows the existence of a function that is Riemann integrable and not Borel measurable). 
\end{itemize}
\end{q}

\begin{q}
\begin{itemize}
\item Construct a sequence of set by $K_1 = [0,1]$ and then $K_2 = [0,1/3]\cup[2/3,1]$ then instead of removing the middle thirds as in the construction of the Cantor set to construct $K_n$ we remove the middle $\alpha_n$ proportion of every interval so that $\lambda(K_n) = (1-\alpha_n) \lambda(K_{n-1})$. Then for $\beta \in (0,1)$ choose $\alpha_n = 1- \beta^{2^{-n}}$, and show that $\bigcap_n K_n$ is a closed set with no interior points and $\lambda(\bigcup_n K_n) = \beta$.
\item Now assume (you can prove this fairly easily but its not the point of the question) that you can find a function $f_n$ which is cotinuous and takes values in $[0,1]$ such that $f_n =0$ on $K_n$ and 1 on $K_{n-1}^c$. Show that $f_n$ is an increasing sequence of functions which converges to $1_U$ where $U = (\bigcap_n K_n)^c$. 
\item Show that $1_U$ is not Riemann integrable.
\end{itemize}
\end{q}
\end{document}