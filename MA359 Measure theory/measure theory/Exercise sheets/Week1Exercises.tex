\documentclass[11pt]{article}

\usepackage{mathtools}
\usepackage{amsmath, amsthm, amsfonts,amssymb}
\usepackage{enumitem}
\usepackage{graphicx}
\usepackage{colortbl}
\usepackage{tikz}
\usepackage[utf8]{inputenc}
\usepackage{esint}
\usepackage{mathrsfs}
\usepackage{subfig,float}
\usepackage[T1]{fontenc}
\usepackage{mathrsfs}  
\usepackage{bbm} 
\usepackage{enumitem}
\usepackage{enumerate}
\usepackage{mathtools}
\usepackage{dsfont}
\newcommand{\set}[1]{\left\{#1\right\}}

\def\grad{\nabla}
\DeclareMathOperator{\dive}{div}
\DeclareMathOperator{\supp}{supp}
\DeclareMathOperator{\essup}{ess\,sup}
\DeclareMathOperator{\Lip}{Lip}
\DeclareMathOperator{\sgn}{sgn}

% Notation for differentials
\def\d{\,\mathrm{d}}
\def\dv{\d v}
\def \ddt{\frac{\mathrm{d}}{\mathrm{d}t}}
\def \ddt{\frac{\mathrm{d}}{\mathrm{d}t}}
\def \ddr{\frac{\mathrm{d}}{\mathrm{d}r}}
\newcommand{\sign}{\text{sign}}
\DeclareMathOperator{\hess}{Hess}

% Generate a PDF with hyperlinks in references.
\usepackage[colorlinks=true,linkcolor=blue,citecolor=blue,urlcolor=blue,breaklinks]{hyperref}

% Bibliography
%-----------------------------------------------------------------

% This uses a bibliography style which hyperlinks the paper titles to
% the paper URL specified in the bibtex file. It also uses natbib,
% which cites papers by name such as Euler (1770) instead of [17].
\usepackage{hyperref}
\usepackage{breakurl}
\usepackage[square,sort,comma,numbers]{natbib}
%\usepackage{natbib}
\usepackage{url}
%\bibliographystyle{plainnat-linked}
\bibliographystyle{plain}

%\usepackage[notcite,notref]{showkeys}
%\usepackage{hyperref}
%\usepackage{breakurl}
\usepackage[square,sort,comma,numbers]{natbib}
%\usepackage{natbib}
%\usepackage{url}
%\usepackage[colorlinks=blue,linkcolor=blue,citecolor=blue,urlcolor=blue,breaklinks]{hyperref}
\bibliographystyle{plain}
\DeclarePairedDelimiter\abs{\lvert}{\rvert}%
\DeclarePairedDelimiter\norm{\lVert}{\rVert}%


\addtolength{\oddsidemargin}{-.875in}
\addtolength{\evensidemargin}{-.875in}
\addtolength{\textwidth}{1.75in}

\addtolength{\topmargin}{-.875in}
\addtolength{\textheight}{1.75in}

% Shortcuts
%-----------------------------------------------------------------

%\newcommand{\abs}[1]{\left\mid#1\right\mid}
%\newcommand{\ap}[1]{\left\langle#1\right\rangle}
%\newcommand{\norm}[1]{\left\mid#1\right\mid}
% \newcommand{\tnorm}[1]{\left\mid\!\left\mid\!\left\mid#1\right\mid\!\right\mid\!\right\mid}

\definecolor{lpink}{rgb}{0.96, 0.76, 0.76}
\definecolor{dpink}{rgb}{0.97, 0.51, 0.47}
\definecolor{sky}{rgb}{0.53, 0.81, 0.92}
\definecolor{salmon}{rgb}{1.0, 0.55, 0.41}
\definecolor{orman}{rgb}{0.24, 0.7, 0.44}
\definecolor{aciksari}{rgb}{0.91, 0.84, 0.42}
\definecolor{dgrey}{rgb}{0.52, 0.52, 0.51}

\def\R{\mathbb{R}}
\def\C{\mathbb{C}}
\def\P{\mathscr{P}}
\def\NN{\mathbb{N}}
\def\Q{\mathbb{Q}}

\def\ird{\int_{\mathbb{R}^N}}
\def\d{\,\mathrm{d}}
\def\dx{\,\mathrm{d}x}
\def\dy{\,\mathrm{d}y}
\def\p{\,\partial}
\newcommand{\en}{\mathcal{H}}
\newcommand{\havva}[1]{{\textcolor{blue}{[\textbf{H:} #1]}}}

% Operators

\def\grad{\nabla}
\def\weakto{\rightharpoonup}

\DeclareMathOperator{\divergence}{div}
%\newcommand{\dv}[1]{\divergence \left(#1\right)}
%
%\DeclareMathOperator{\supp}{supp}
%\DeclareMathOperator{\essup}{ess\,sup}
%\DeclareMathOperator{\Lip}{Lip}
\DeclareMathOperator{\law}{law}

\DeclareMathOperator{\Pf}{Pf.}
\DeclareMathOperator{\Fp}{Fp.}
\DeclareMathOperator{\pv}{pv.}

\newcommand{\for}{\quad \text{ for all }}
\newcommand{\tb}[1]{\textcolor{blue}{#1}}


\newcommand{\OmT}{\Omega\times (0,T)}
\let\pa\partial
\let\na\nabla
\newcommand{\red}[1]{\textcolor{red}{#1}}
\newcommand{\cD}{\mathcal{D}}
\let\eps\varepsilon
\newcommand{\wen}{w^{(\varepsilon,N)}}
\newcommand{\OmTc}{\overline{\Omega}\times [0,T]}
\newcommand{\rrhoA}{\sqrt{\rho_A}}
\newcommand{\rrhoB}{\sqrt{\rho_B}}
\newcommand{\rrhoAB}{\sqrt{\rho_A\rho_B}}

% Theorems
%-----------------------------------------------------------------
\newtheorem{thm}{Theorem}[section]
% \newtheorem{thm}{Theorem}
\newtheorem{cor}[thm]{Corollary}
\newtheorem{lem}[thm]{Lemma}
\newtheorem{prp}[thm]{Proposition}
\newtheorem{claim}[thm]{Claim}
% \newtheorem{hyp}[thm]{Hypothesis}
\newtheorem{hyp}{Hypothesis}
\theoremstyle{definition}
\newtheorem{dfn}[thm]{Definition}
\theoremstyle{remark}
\newtheorem{remark}[thm]{Remark}
\newtheorem{ex}[thm]{Example}
\newtheorem{q}{Question}

\hypersetup{pdftitle={Measure Theory}}
\hypersetup{pdfauthor={Josephine Evans }}


\author{
Josephine Evans
}
\title{Measure Theory: Exercises (not for credit)}
%\date{}

\makeindex

\begin{document}
\maketitle

\begin{q}
Find the $\sigma$-algebra on $\mathbb{R}$ which is generated by the collection of all one-point sets.
\end{q}

\begin{q}
Find an example to show that the union of a collection of $\sigma$-algebras is not necessarilly a $\sigma$-algebra.
\end{q}

\begin{q}
Prove that if $\mathcal{E}$ is both a $d$-system and $\pi$-system then it is a $\sigma$-algebra. Use this to prove Dynkin's $\pi$-system lemma that if $\mathcal{A}$ is a $\pi$-system then any $d$-system containing $\mathcal{A}$ also contains $\sigma(A)$. \emph{Hint: Consider $\mathcal{D}$ the intersection of all $d$-systems containing $\mathcal{A}$, and $\mathcal{D}' = \{B \in \mathcal{D}\,:\, B \cap A \in \mathcal{D} \, \forall \, A \in \mathcal{A}\}$ and $\mathcal{D}'' = \{ B \in \mathcal{D} \, :\, B \cap A \in \mathcal{D} \, \forall A \in \mathcal{D}\}$ and show that they are all $d$-systems.}
\end{q}

\begin{q}
Suppose that $(E, \mathcal{E}, \mu)$ is a measure space. Prove the inclusion-exclusion formula
\[ \mu(A_1 \cup A_2 \cup \dots A_n) = \sum_{k=1}^n \mu(A_k)- \sum_{k \neq j}^n \mu(A_k \cap A_j) + \dots + (-1)^n \mu(A_1 \cap A_2 \cap \dots A_n). \]
\end{q}

\begin{q}
Let $\mu$ be the measure on $\mathbb{R}$ defined by setting $\mu(A)$ to be the number of rationals in the set $A$ (where $\mu(A) = \infty$ if there are infinitely many rationals). Show that $\mu$ is a $\sigma$-finite measure which gives every open interval infinite measure. 
\end{q}

\begin{q}
Let $\mu$ be a finitely additive set function on a $\sigma$-algebra, $\mathcal{E}$ in a set $E$ with $\mu(E) < \infty$. Show that $\mu$ is countably additive \emph{if and only if} for any decreasing sequence of sets $A_n$ with $\bigcap_n A_n = \emptyset$ and $\mu(A_1) < \infty$ then we have $\mu(A_n) \rightarrow 0$.
\end{q}

\begin{q}
Let $(E, \mathcal{E}, \mu)$ be a measure space. We call a set $N$ a \emph{null set} if there exists $B \in \mathcal{E}$ with $N \subseteq B$ and $\mu(B) = 0$. We write $\mathcal{N}$ for the collection of all null sets. Define the collection
\[ \mathcal{E}^\mu = \{ A \cup N \, :\, A \in \mathcal{E}, N \in \mathcal{N}\}. \] Show that $\mathcal{E}^\mu$ is a $\sigma$-algebra and the extension of $\mu$ to $\mathcal{E}^\mu$ defined by $\mu(A \cup N) = \mu(A)$ is a measure. We call $\mathcal{E}^\mu$ the \emph{completion of $\mathcal{E}$ with respect to $\mu$}.
\end{q}
\end{document}