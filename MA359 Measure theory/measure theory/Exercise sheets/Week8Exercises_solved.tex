\documentclass[11pt]{article}

\usepackage{mathtools}
\usepackage{amsmath, amsthm, amsfonts,amssymb}
\usepackage{enumitem}
\usepackage{graphicx}
\usepackage{colortbl}
\usepackage{tikz}
\usepackage[utf8]{inputenc}
\usepackage{esint}
\usepackage{mathrsfs}
\usepackage{subfig,float}
\usepackage[T1]{fontenc}
\usepackage{mathrsfs}  
\usepackage{bbm} 
\usepackage{enumitem}
\usepackage{enumerate}
\usepackage{mathtools}
\usepackage{dsfont}
\newcommand{\set}[1]{\left\{#1\right\}}

\def\grad{\nabla}
\DeclareMathOperator{\dive}{div}
\DeclareMathOperator{\supp}{supp}
\DeclareMathOperator{\essup}{ess\,sup}
\DeclareMathOperator{\Lip}{Lip}
\DeclareMathOperator{\sgn}{sgn}

% Notation for differentials
\def\d{\,\mathrm{d}}
\def\dv{\d v}
\def \ddt{\frac{\mathrm{d}}{\mathrm{d}t}}
\def \ddt{\frac{\mathrm{d}}{\mathrm{d}t}}
\def \ddr{\frac{\mathrm{d}}{\mathrm{d}r}}
\newcommand{\sign}{\text{sign}}
\DeclareMathOperator{\hess}{Hess}

% Generate a PDF with hyperlinks in references.
\usepackage[colorlinks=true,linkcolor=blue,citecolor=blue,urlcolor=blue,breaklinks]{hyperref}

% Bibliography
%-----------------------------------------------------------------

% This uses a bibliography style which hyperlinks the paper titles to
% the paper URL specified in the bibtex file. It also uses natbib,
% which cites papers by name such as Euler (1770) instead of [17].
\usepackage{hyperref}
\usepackage{breakurl}
\usepackage[square,sort,comma,numbers]{natbib}
%\usepackage{natbib}
\usepackage{url}
%\bibliographystyle{plainnat-linked}
\bibliographystyle{plain}

%\usepackage[notcite,notref]{showkeys}
%\usepackage{hyperref}
%\usepackage{breakurl}
\usepackage[square,sort,comma,numbers]{natbib}
%\usepackage{natbib}
%\usepackage{url}
%\usepackage[colorlinks=blue,linkcolor=blue,citecolor=blue,urlcolor=blue,breaklinks]{hyperref}
\bibliographystyle{plain}
\DeclarePairedDelimiter\abs{\lvert}{\rvert}%
\DeclarePairedDelimiter\norm{\lVert}{\rVert}%


\addtolength{\oddsidemargin}{-.875in}
\addtolength{\evensidemargin}{-.875in}
\addtolength{\textwidth}{1.75in}

\addtolength{\topmargin}{-.875in}
\addtolength{\textheight}{1.75in}

% Shortcuts
%-----------------------------------------------------------------

%\newcommand{\abs}[1]{\left\mid#1\right\mid}
%\newcommand{\ap}[1]{\left\langle#1\right\rangle}
%\newcommand{\norm}[1]{\left\mid#1\right\mid}
% \newcommand{\tnorm}[1]{\left\mid\!\left\mid\!\left\mid#1\right\mid\!\right\mid\!\right\mid}

\definecolor{lpink}{rgb}{0.96, 0.76, 0.76}
\definecolor{dpink}{rgb}{0.97, 0.51, 0.47}
\definecolor{sky}{rgb}{0.53, 0.81, 0.92}
\definecolor{salmon}{rgb}{1.0, 0.55, 0.41}
\definecolor{orman}{rgb}{0.24, 0.7, 0.44}
\definecolor{aciksari}{rgb}{0.91, 0.84, 0.42}
\definecolor{dgrey}{rgb}{0.52, 0.52, 0.51}

\def\R{\mathbb{R}}
\def\C{\mathbb{C}}
\def\P{\mathscr{P}}
\def\NN{\mathbb{N}}
\def\Q{\mathbb{Q}}

\def\ird{\int_{\mathbb{R}^N}}
\def\d{\,\mathrm{d}}
\def\dx{\,\mathrm{d}x}
\def\dy{\,\mathrm{d}y}
\def\p{\,\partial}
\newcommand{\en}{\mathcal{H}}
\newcommand{\havva}[1]{{\textcolor{blue}{[\textbf{H:} #1]}}}

% Operators

\def\grad{\nabla}
\def\weakto{\rightharpoonup}

\DeclareMathOperator{\divergence}{div}
%\newcommand{\dv}[1]{\divergence \left(#1\right)}
%
%\DeclareMathOperator{\supp}{supp}
%\DeclareMathOperator{\essup}{ess\,sup}
%\DeclareMathOperator{\Lip}{Lip}
\DeclareMathOperator{\law}{law}

\DeclareMathOperator{\Pf}{Pf.}
\DeclareMathOperator{\Fp}{Fp.}
\DeclareMathOperator{\pv}{pv.}

\newcommand{\for}{\quad \text{ for all }}
\newcommand{\tb}[1]{\textcolor{blue}{#1}}


\newcommand{\OmT}{\Omega\times (0,T)}
\let\pa\partial
\let\na\nabla
\newcommand{\red}[1]{\textcolor{red}{#1}}
\newcommand{\cD}{\mathcal{D}}
\let\eps\varepsilon
\newcommand{\wen}{w^{(\varepsilon,N)}}
\newcommand{\OmTc}{\overline{\Omega}\times [0,T]}
\newcommand{\rrhoA}{\sqrt{\rho_A}}
\newcommand{\rrhoB}{\sqrt{\rho_B}}
\newcommand{\rrhoAB}{\sqrt{\rho_A\rho_B}}

% Theorems
%-----------------------------------------------------------------
\newtheorem{thm}{Theorem}[section]
% \newtheorem{thm}{Theorem}
\newtheorem{cor}[thm]{Corollary}
\newtheorem{lem}[thm]{Lemma}
\newtheorem{prp}[thm]{Proposition}
\newtheorem{claim}[thm]{Claim}
% \newtheorem{hyp}[thm]{Hypothesis}
\newtheorem{hyp}{Hypothesis}
\theoremstyle{definition}
\newtheorem{dfn}[thm]{Definition}
\theoremstyle{remark}
\newtheorem{remark}[thm]{Remark}
\newtheorem{ex}[thm]{Example}
\newtheorem{q}{Question}
\newenvironment{ans}{\paragraph{Answer:}}{\hfill$\square$\vspace{10pt}}

\hypersetup{pdftitle={Measure Theory}}
\hypersetup{pdfauthor={Josephine Evans }}


\author{
Josephine Evans
}
\title{Measure Theory: Exercises (not for credit)}
%\date{}

\makeindex

\begin{document}
\maketitle

\begin{q}
Use the inequality $(x-y)^2 \geq 0$ to show that $xy \leq (x^2 + y^2)/2$.
\end{q}
\begin{ans}
$(x-y)^2 \geq 0$ turns into $x^2 +y^2 -2xy \geq 0$ so rearranging gives the inequality.
\end{ans}

\begin{q}
Draw a graph of the function $t=s^{p-1}$. Let $L$ be the are underneath the graph and above the $s$-axis and $U$ be the area above the graph and to the right of the $t$-axis. Compute the areas of $L \cap \{s\,:\, 0\leq s \leq x\}$ and $U \cap \{t\,:\, 0 \leq t \leq x\}$ and draw these two sets on your picture. Use this to show that $xy \leq x^p/p +y^q/q$.
\end{q}
\begin{ans}
It is hard to type an answer to this but the area under the graph will have size $x^p/p$ and the area above the graph will have size $y^q/q$ and the box of width $x$ and height $y$ is contained inside the union of the two areas.
\end{ans}

\begin{q}
Let $(E,\mathcal{E}, \mu)$ be a finite measure space and let $p_1 \leq p_2$ show that if $f \in L^{p_2}$ then $f \in L^{p_1}$.
\end{q}
\begin{ans}
You can do this using either Holder or Jensen's inequality. The method via Jensen is
\[ \left(\int_E |f(x)|^{p_1} \mu(\mathrm{dx})\right)^{p_2/p_1} = \mu(E)^{p_1/p_2} \left( \frac{1}{\mu(E)} \int_E |f(x)|^{p_1} \mu(\mathrm{d}x) \right)^{p_2/p_1}.  \] Then the measure $\mu/\mu(E)$ gives the space $E$ measure 1, and $x \mapsto x^{p_2/p_1}$ is a convex function so we can apply Jensen to get
\[\mu(E)^{p_1/p_2} \left( \frac{1}{\mu(E)} \int_E |f(x)|^{p_1} \mu(\mathrm{d}x) \right)^{p_2/p_1} \leq \mu(E)^{p_1/p_2} \frac{1}{\mu(E)} \int |f(x)|^{p_1 * p_2/p_1} \mu(\mathrm{d}x) = \mu(E)^{p_1/p_2-1} \|f\|_{p_2}^{p_2}. \] Putting this all together we have
\[ \|f\|_{p_1} \leq \mu(E)^{p_1/p_2 -1} \|f\|_{p_2}. \]
\end{ans}

\begin{q}
Let $X$ be a random variable. Prove the identity
\[ \mathbb{E}(|X|^p) = \int_0^\infty p x^{p-1} \mathbb{P}(|X| > x) \mathrm{d}x \] and hence show that if for all $q >p$ we have $\mathbb{P}(|X| > x) = O(x^{-q})$ as $x \rightarrow \infty$ then $X \in L^p$. (Recall that $X$ is a measurable function from a probability space to $\mathbb{R}$, $\mathbb{P}$ is the measure on this space and $\mathbb{E}$ is the notation for integrating with respect to $\mathbb{P}$.
\end{q}
\begin{ans}
\begin{align*}
\mathbb{E}(|X|^p) & = \int_\Omega |x|^p \mathbb{P}(\mathrm{d}x) = \int_\Omega \left( \int_0^{|x|} p y^{p-1} \mathrm{d}y \right) \mathbb{P}(\mathrm{d}x) \\
& = \int_{\Omega} \int_0^\infty py^{p-1} 1_{y \leq |x|} \mathrm{d}y \mathbb{P}(\mathrm{d}x) \\
&= \int_0^\infty \int_\Omega 1_{y \leq |x|} \mathbb{P}( \mathrm{d}x) py^{p-1} \mathrm{d}y\\
& = \int_0^\infty \mathbb{P}(|X| \geq y) p y^{p-1} \mathrm{d}y.
\end{align*}
Now if $\mathbb{P}(|X| > x) = O(x^{-q})$ for every $q>p$ then there will exist a $C$ such that $\mathbb{P}(|X| > x) \leq Cx^{-p-1}$ then
\[ \mathbb{P}(|X|^p) \leq  \int_1^\infty \min\{ py^{p-1}, Cy^{-2}\} \mathrm{d}y < \infty. \]
\end{ans}
\end{document}