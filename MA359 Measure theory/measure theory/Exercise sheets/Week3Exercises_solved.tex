\documentclass[11pt]{article}

\usepackage{mathtools}
\usepackage{amsmath, amsthm, amsfonts,amssymb}
\usepackage{enumitem}
\usepackage{graphicx}
\usepackage{colortbl}
\usepackage{tikz}
\usepackage[utf8]{inputenc}
\usepackage{esint}
\usepackage{mathrsfs}
\usepackage{subfig,float}
\usepackage[T1]{fontenc}
\usepackage{mathrsfs}  
\usepackage{bbm} 
\usepackage{enumitem}
\usepackage{enumerate}
\usepackage{mathtools}
\usepackage{dsfont}
\newcommand{\set}[1]{\left\{#1\right\}}

\def\grad{\nabla}
\DeclareMathOperator{\dive}{div}
\DeclareMathOperator{\supp}{supp}
\DeclareMathOperator{\essup}{ess\,sup}
\DeclareMathOperator{\Lip}{Lip}
\DeclareMathOperator{\sgn}{sgn}

% Notation for differentials
\def\d{\,\mathrm{d}}
\def\dv{\d v}
\def \ddt{\frac{\mathrm{d}}{\mathrm{d}t}}
\def \ddt{\frac{\mathrm{d}}{\mathrm{d}t}}
\def \ddr{\frac{\mathrm{d}}{\mathrm{d}r}}
\newcommand{\sign}{\text{sign}}
\DeclareMathOperator{\hess}{Hess}

% Generate a PDF with hyperlinks in references.
\usepackage[colorlinks=true,linkcolor=blue,citecolor=blue,urlcolor=blue,breaklinks]{hyperref}

% Bibliography
%-----------------------------------------------------------------

% This uses a bibliography style which hyperlinks the paper titles to
% the paper URL specified in the bibtex file. It also uses natbib,
% which cites papers by name such as Euler (1770) instead of [17].
\usepackage{hyperref}
\usepackage{breakurl}
\usepackage[square,sort,comma,numbers]{natbib}
%\usepackage{natbib}
\usepackage{url}
%\bibliographystyle{plainnat-linked}
\bibliographystyle{plain}

%\usepackage[notcite,notref]{showkeys}
%\usepackage{hyperref}
%\usepackage{breakurl}
\usepackage[square,sort,comma,numbers]{natbib}
%\usepackage{natbib}
%\usepackage{url}
%\usepackage[colorlinks=blue,linkcolor=blue,citecolor=blue,urlcolor=blue,breaklinks]{hyperref}
\bibliographystyle{plain}
\DeclarePairedDelimiter\abs{\lvert}{\rvert}%
\DeclarePairedDelimiter\norm{\lVert}{\rVert}%


\addtolength{\oddsidemargin}{-.875in}
\addtolength{\evensidemargin}{-.875in}
\addtolength{\textwidth}{1.75in}

\addtolength{\topmargin}{-.875in}
\addtolength{\textheight}{1.75in}

% Shortcuts
%-----------------------------------------------------------------

%\newcommand{\abs}[1]{\left\mid#1\right\mid}
%\newcommand{\ap}[1]{\left\langle#1\right\rangle}
%\newcommand{\norm}[1]{\left\mid#1\right\mid}
% \newcommand{\tnorm}[1]{\left\mid\!\left\mid\!\left\mid#1\right\mid\!\right\mid\!\right\mid}

\definecolor{lpink}{rgb}{0.96, 0.76, 0.76}
\definecolor{dpink}{rgb}{0.97, 0.51, 0.47}
\definecolor{sky}{rgb}{0.53, 0.81, 0.92}
\definecolor{salmon}{rgb}{1.0, 0.55, 0.41}
\definecolor{orman}{rgb}{0.24, 0.7, 0.44}
\definecolor{aciksari}{rgb}{0.91, 0.84, 0.42}
\definecolor{dgrey}{rgb}{0.52, 0.52, 0.51}

\def\R{\mathbb{R}}
\def\C{\mathbb{C}}
\def\P{\mathscr{P}}
\def\NN{\mathbb{N}}
\def\Q{\mathbb{Q}}

\def\ird{\int_{\mathbb{R}^N}}
\def\d{\,\mathrm{d}}
\def\dx{\,\mathrm{d}x}
\def\dy{\,\mathrm{d}y}
\def\p{\,\partial}
\newcommand{\en}{\mathcal{H}}
\newcommand{\havva}[1]{{\textcolor{blue}{[\textbf{H:} #1]}}}

% Operators

\def\grad{\nabla}
\def\weakto{\rightharpoonup}

\DeclareMathOperator{\divergence}{div}
%\newcommand{\dv}[1]{\divergence \left(#1\right)}
%
%\DeclareMathOperator{\supp}{supp}
%\DeclareMathOperator{\essup}{ess\,sup}
%\DeclareMathOperator{\Lip}{Lip}
\DeclareMathOperator{\law}{law}

\DeclareMathOperator{\Pf}{Pf.}
\DeclareMathOperator{\Fp}{Fp.}
\DeclareMathOperator{\pv}{pv.}

\newcommand{\for}{\quad \text{ for all }}
\newcommand{\tb}[1]{\textcolor{blue}{#1}}


\newcommand{\OmT}{\Omega\times (0,T)}
\let\pa\partial
\let\na\nabla
\newcommand{\red}[1]{\textcolor{red}{#1}}
\newcommand{\cD}{\mathcal{D}}
\let\eps\varepsilon
\newcommand{\wen}{w^{(\varepsilon,N)}}
\newcommand{\OmTc}{\overline{\Omega}\times [0,T]}
\newcommand{\rrhoA}{\sqrt{\rho_A}}
\newcommand{\rrhoB}{\sqrt{\rho_B}}
\newcommand{\rrhoAB}{\sqrt{\rho_A\rho_B}}

% Theorems
%-----------------------------------------------------------------
\newtheorem{thm}{Theorem}[section]
% \newtheorem{thm}{Theorem}
\newtheorem{cor}[thm]{Corollary}
\newtheorem{lem}[thm]{Lemma}
\newtheorem{prp}[thm]{Proposition}
\newtheorem{claim}[thm]{Claim}
% \newtheorem{hyp}[thm]{Hypothesis}
\newtheorem{hyp}{Hypothesis}
\theoremstyle{definition}
\newtheorem{dfn}[thm]{Definition}
\theoremstyle{remark}
\newtheorem{remark}[thm]{Remark}
\newtheorem{ex}[thm]{Example}
\newtheorem{q}{Question}
\newenvironment{ans}{\paragraph{Answer:}}{\hfill$\square$}

\hypersetup{pdftitle={Measure Theory}}
\hypersetup{pdfauthor={Josephine Evans }}


\author{
Josephine Evans
}
\title{Measure Theory: Exercises (not for credit)}
%\date{}

\makeindex

\begin{document}
\maketitle

\begin{q}
Use the uniqueness of extension theorem from the notes to show that Lebesgue measure in $\mathbb{R}^d$ is translation invariant on $\mathcal{B}(\mathbb{R})$. That is to say if we define the set $A+x = \{z \in \mathbb{R}^d\,:\, z=x+y, y \in A \}$. Show also that Lebesgue measure in $\mathbb{R}^d$ is rotationally invariant. That is to say if $M$ is a rotation matrix in $\mathbb{R}^d$ then $\lambda(A) = \lambda(M^{-1}(A))$ where we understand $M$ here to represent the map $x \mapsto Mx$.
\end{q}
\begin{ans}
This is done in the lecture notes in $\mathbb{R}$ rather than $\mathbb{R}^d$. The proof is essentially the same. The rectangles $R= (a_1, b_1] \times (a_2, b_2] \times \dots \times (a_d, b_d]$ are a $\pi$-system generating the Borel $\sigma$-algebra on $\mathbb{R}^d$ (they essentially prove this in the assignment). Define a measure by $\lambda_x(A) = \lambda(A+x)$. Then we want to check that $\lambda_x$ agrees with $\lambda$ on the rectangles. A rectangle shifted by $x$ is still a rectangle and so we can calculate their $d$-dimensional volume and see that $\lambda(R) = \lambda(x +R)$ for any $R$. The uniqueness of extension lemma gives us that $\lambda_x = \lambda$ on every Borel set. 

Let $M_\theta$ be the map that rotates everything clockwise about the origin with angle $\theta$. Now we want to show that $\lambda(M_\theta(R)) = \lambda(R)$ and the result in lectures will complete the proof. In order to do this fixe $\epsilon >0$ and then choose some small $\delta$ such that $(0, delta]^d$ is contained inside the ball of radius $\epsilon$ centred at $(\delta/2, \dots, \delta/2)$. Then tile the whole of $\mathbb{R}^d$ with non-overlapping coppies of the little cube or radius $\delta$ (here it is useful that we are using half open rectangles rather than open ones as it allows us to cover the whole of $\mathbb{R}^d$. Its a bit difficult to write how to do this but for every $\mathbf{n} \in \mathbb{Z}^d$ we include the tile $\bigcup_{\mathbf{n} \in \mathbb{Z}^d}$. Then define $R_1$ to be the union of all the little squares that intersect with $M_\theta(R)$ and $R_2$ to be the union of all the little squares which are completely included in $M_\theta(R)$. Then since the little squares entirely covered $\mathbb{R}^d$ we have $R_2 \subseteq M_\theta(R) \subseteq R_1$. By monotonicity of Lebesgue measure we have that $\lambda(R_2) \leq \lambda(M_\theta(R)) \leq \lambda(R_1)$. We can compute the areas of $R_1$ and $R_2$ from now but its fiddly and long. We haven't forgotten everything about how normal volume works so and $R_1$ and $R_2$ are \textbf{finite, disjoint} unions of little cubes, so we can define two larger rotated rectangles $\tilde{R_1}$ and $\tilde{R_2}$ by either shrinking each side of $M_\theta(R)$ by $\epsilon$ or growing each side of $M_\theta(R)$ by $\epsilon$ and say that
\[ vol(\tilde{R}_2) \leq vol(R_2) = \lambda(R_2) \leq \lambda(M_\theta(R)) \leq \lambda(R_1)=vol(R_1) \leq vol(\tilde{R}_2), \] where we are able to equate the volumes of $R_i$ with their lebesgue measures as they are finite disjoint unions of half open rectangles. This is really just a convinient method for summing up the areas of all the little cubes in $R_1$ and $R_2$. If working with normal volumes makes you nervous you can work out by hand how to sum up the areas of all the little rectangles. However, it is quite an important point that we need to be able to understand the volume of finite disjoint unions of rectangles in order to make sense of Lebesgue measure. There is nothing unrigorous about working with volumes of rectangles in the way you would have before knowing about Lebesgue measure (whatever angle they are at) as long as you only have finitely many!
\end{ans}

\begin{q}
Let $\lambda$ be Lebesgue measure on $\mathbb{R}^d$ let $L_{m,c} = \{ (x,y)\,:\, y=mx+c \}$ for some $m, c \in \mathbb{R}$. Show that $\lambda{L}=0$.
\end{q}
\begin{ans}
There are numerous ways of doing this essentially we need to show that we can cover the line with rectangles whose total mass will end up being very small. First lets look at what happens when $m = c= 0$ (i.e. we look at the measure of the $x$-axis).For any $\epsilon$ we can cover this with a countable union of rectangles that look like $[n,n+1) \times [-\epsilon 2^{-n}, \epsilon 2^{-n})$ and $[-n-1, -n) \times  [-\epsilon 2^{-n}, \epsilon 2^{-n})$. Summing up the area of all these rectangles gives $\lambda^*(L_{0,0}) \leq 2\epsilon$. As $\epsilon$ is arbitrary this shows us that $\lambda(L_{0,0})=0$. By translation invariance of Lebesgue measure we can see that $\lambda(L_{m,c}) = \lambda(L_{m,0})$. We then also show that Lebesge measure is rotationally invariant this is similar to what we would have to do in order to show that the lines have measure 0 but it is a useful extension Therefore we have $\lambda(L_{m,0}) = \lambda(L_{0,0}) = 0$.
\end{ans}

\begin{q}
Let $f$ be the map on $\mathbb{R}^d$ given by $f(x) = 3x$. Write down an expression for $\lambda(f(A))$ and prove that it is correct. 
\end{q}
\begin{ans}
Let us write $\mu(A) = \lambda(f(A))$ then we expect that $\mu(A) = 3^d \lambda(A)$. We can then show that $\mu$ and $3^d\lambda$ agree on the rectangles in $\mathbb{R}^d$. Therefore by uniqueness of extension they agree everywhere.
\end{ans}

\begin{q}
Prove that if $A$ is Lebesgue measurable then $A= B \cup N$ where $B$ is an $F_\sigma$ set (a countable union of closed sets) and $N$ is a null set. Show the converse, that if $A = B \cup N$ where $B$ is $F_\sigma$ and $N$ is null then $A$ is Lebesgue measurable.
\end{q}
\begin{ans}
First let us suppose that $\lambda(A) < \infty$. By our proposition on the regularity of Lebesgue measure we know that $\lambda(A) = \sup \{ \lambda(K)\,:\, \mbox{$K$ is compact}, K \subseteq A\}$. Therefore, for each $n$ there exists $K_n \subseteq A$ with $\lambda(K_n) \geq \lambda(A) - 1/n$. Write $B= \bigcup_n K_n$. Since the sets $K$ are compact they are closed, therefore $B$ is an $F_\sigma$ set. We also have that $K_n \subseteq A$. As $B$ is $F_\sigma$ it is a Borel set and therefore is Lebesgue measurable. Therefore $A \setminus B$ is also a Lebesgue measureable set. $\lambda(A \setminus B) = \lambda(A) - \lambda(B)$ by countable (or just finite) additivity. We also have (by continuity of measure) that $\lambda(B) = \lim_n \bigcup_{k=1}^n K_k$ and $\lambda(\bigcup_{k=1}^n K_k) \geq \lambda(K_n) \geq \lambda(A) - 1/n$ and $\lambda(B) \leq \lambda(A)$ by monotonicity. Therefore $\lambda(A) = \lambda(B)$ so $A\setminus B$ is a null set. This works if $\mu(A)< \infty$ but if it isn't we could be subtracting infinity from infinity. To get round this we do our usual trick of writing $A_n = A \cap B(0,n)$ where $B(0,n)$ is the closed ball of radius $n$ centre $0$. Then $A = \bigcup_n A_n$. For each $n$ we can construct $B_n$ and $N_n$ as above and since the countable union of $F_\sigma$ sets is also $F_\sigma$ and the countable union of null sets is null we just choose $B= \bigcup_n B_n$ and $N= \bigcup_n N_n$ This proves our result.

To prove the converse, we already know that any null set $N$ is Lebesgue measurable by a result on the first assignment. We also know that any Borel set $B$ is Lebesgue measurable. Since the Lebesgue measurable sets form a $\sigma$-algebra this shows that $B \cup N$ is also Lebesgue measurable.
\end{ans}

\begin{q}
Show that there is a Lebesgue measurable subset of $\mathbb{R}^2$ whose projection under the map $(x,y) \mapsto x$ is not Lebesgue measurable.
\end{q}
\begin{ans}
Take $A$ a non-Lebesgue measurable subset of $\mathbb{R}$ and define $B = \{(x,x) \,:\, x \in A\}$ then $B$ is contained inside the line $y=x$ and is therefore a null set. Since all null sets are Lebesgue measurable $B$ is Lebesgue measurable. Howevery $B \mapsto A$ under $(x,y) \mapsto x$ and $A$ is not Lebesgue measurable.
\end{ans}

\begin{q}
Let $E,F$ be topological spaces and let $f: E \rightarrow F$ be a continuous function. Show that $f$ is measurable with respect to the Borel $\sigma$ algebras on $E$ and $F$.
\end{q}
\begin{ans}
The open sets are a collection of subsets generating the Borel $\sigma$-algebra on both $E$ and $F$. Therefore by a result given in lecture (Lemma 2.2 week 3 notes) if $f^{-1}(U) \in \mathcal{B}(E)$ for every open $U \subseteq F$ then $f$ is measurable. As $f$ is continuous $f^{-1}(U)$ is open, so it is in $\mathcal{B}(E)$ and we are done.
\end{ans}

\begin{q}
Let $(E, \mathcal{A})$ be a measurable space and let $A \in \mathcal{A}$. Show that $1_A$ the indicator function of $A$ is measurable. 
\end{q}
\begin{ans}
Let $f(x) = 1_{A}(x)$ then $f^{-1}(B) = E$ if $\{0,1\} \in B$, $f^{-1}(B) = A$ if $\{1\} \in B, \{0\} \notin B$ and $f^{-1}(B) = A^c$ if $\{1\} \notin B, \{0\} \in B$ and $f^{-1}(B) = \emptyset$ if $\{0,1\} \notin B$. Since $E, \emptyset, A, A^c$ are all measurable then $f$ is measurable.
\end{ans}




\end{document}