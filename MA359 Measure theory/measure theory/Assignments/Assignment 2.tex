\documentclass[11pt]{article}

\usepackage{mathtools}
\usepackage{amsmath, amsthm, amsfonts,amssymb}
\usepackage{enumitem}
\usepackage{graphicx}
\usepackage{colortbl}
\usepackage{tikz}
\usepackage[utf8]{inputenc}
\usepackage{esint}
\usepackage{mathrsfs}
\usepackage{subfig,float}
\usepackage[T1]{fontenc}
\usepackage{mathrsfs}  
\usepackage{bbm} 
\usepackage{enumitem}
\usepackage{enumerate}
\usepackage{mathtools}
\usepackage{dsfont}
\newcommand{\set}[1]{\left\{#1\right\}}

\def\grad{\nabla}
\DeclareMathOperator{\dive}{div}
\DeclareMathOperator{\supp}{supp}
\DeclareMathOperator{\essup}{ess\,sup}
\DeclareMathOperator{\Lip}{Lip}
\DeclareMathOperator{\sgn}{sgn}

% Notation for differentials
\def\d{\,\mathrm{d}}
\def\dv{\d v}
\def \ddt{\frac{\mathrm{d}}{\mathrm{d}t}}
\def \ddt{\frac{\mathrm{d}}{\mathrm{d}t}}
\def \ddr{\frac{\mathrm{d}}{\mathrm{d}r}}
\newcommand{\sign}{\text{sign}}
\DeclareMathOperator{\hess}{Hess}

% Generate a PDF with hyperlinks in references.
\usepackage[colorlinks=true,linkcolor=blue,citecolor=blue,urlcolor=blue,breaklinks]{hyperref}

% Bibliography
%-----------------------------------------------------------------

% This uses a bibliography style which hyperlinks the paper titles to
% the paper URL specified in the bibtex file. It also uses natbib,
% which cites papers by name such as Euler (1770) instead of [17].
\usepackage{hyperref}
\usepackage{breakurl}
\usepackage[square,sort,comma,numbers]{natbib}
%\usepackage{natbib}
\usepackage{url}
%\bibliographystyle{plainnat-linked}
\bibliographystyle{plain}

%\usepackage[notcite,notref]{showkeys}
%\usepackage{hyperref}
%\usepackage{breakurl}
\usepackage[square,sort,comma,numbers]{natbib}
%\usepackage{natbib}
%\usepackage{url}
%\usepackage[colorlinks=blue,linkcolor=blue,citecolor=blue,urlcolor=blue,breaklinks]{hyperref}
\bibliographystyle{plain}
\DeclarePairedDelimiter\abs{\lvert}{\rvert}%
\DeclarePairedDelimiter\norm{\lVert}{\rVert}%


\addtolength{\oddsidemargin}{-.875in}
\addtolength{\evensidemargin}{-.875in}
\addtolength{\textwidth}{1.75in}

\addtolength{\topmargin}{-.875in}
\addtolength{\textheight}{1.75in}

% Shortcuts
%-----------------------------------------------------------------

%\newcommand{\abs}[1]{\left\mid#1\right\mid}
%\newcommand{\ap}[1]{\left\langle#1\right\rangle}
%\newcommand{\norm}[1]{\left\mid#1\right\mid}
% \newcommand{\tnorm}[1]{\left\mid\!\left\mid\!\left\mid#1\right\mid\!\right\mid\!\right\mid}

\definecolor{lpink}{rgb}{0.96, 0.76, 0.76}
\definecolor{dpink}{rgb}{0.97, 0.51, 0.47}
\definecolor{sky}{rgb}{0.53, 0.81, 0.92}
\definecolor{salmon}{rgb}{1.0, 0.55, 0.41}
\definecolor{orman}{rgb}{0.24, 0.7, 0.44}
\definecolor{aciksari}{rgb}{0.91, 0.84, 0.42}
\definecolor{dgrey}{rgb}{0.52, 0.52, 0.51}

\def\R{\mathbb{R}}
\def\C{\mathbb{C}}
\def\P{\mathscr{P}}
\def\NN{\mathbb{N}}
\def\Q{\mathbb{Q}}

\def\ird{\int_{\mathbb{R}^N}}
\def\d{\,\mathrm{d}}
\def\dx{\,\mathrm{d}x}
\def\dy{\,\mathrm{d}y}
\def\p{\,\partial}
\newcommand{\en}{\mathcal{H}}
\newcommand{\havva}[1]{{\textcolor{blue}{[\textbf{H:} #1]}}}

% Operators

\def\grad{\nabla}
\def\weakto{\rightharpoonup}

\DeclareMathOperator{\divergence}{div}
%\newcommand{\dv}[1]{\divergence \left(#1\right)}
%
%\DeclareMathOperator{\supp}{supp}
%\DeclareMathOperator{\essup}{ess\,sup}
%\DeclareMathOperator{\Lip}{Lip}
\DeclareMathOperator{\law}{law}

\DeclareMathOperator{\Pf}{Pf.}
\DeclareMathOperator{\Fp}{Fp.}
\DeclareMathOperator{\pv}{pv.}

\newcommand{\for}{\quad \text{ for all }}
\newcommand{\tb}[1]{\textcolor{blue}{#1}}


\newcommand{\OmT}{\Omega\times (0,T)}
\let\pa\partial
\let\na\nabla
\newcommand{\red}[1]{\textcolor{red}{#1}}
\newcommand{\cD}{\mathcal{D}}
\let\eps\varepsilon
\newcommand{\wen}{w^{(\varepsilon,N)}}
\newcommand{\OmTc}{\overline{\Omega}\times [0,T]}
\newcommand{\rrhoA}{\sqrt{\rho_A}}
\newcommand{\rrhoB}{\sqrt{\rho_B}}
\newcommand{\rrhoAB}{\sqrt{\rho_A\rho_B}}

% Theorems
%-----------------------------------------------------------------
\newtheorem{thm}{Theorem}[section]
% \newtheorem{thm}{Theorem}
\newtheorem{cor}[thm]{Corollary}
\newtheorem{lem}[thm]{Lemma}
\newtheorem{prp}[thm]{Proposition}
\newtheorem{claim}[thm]{Claim}
% \newtheorem{hyp}[thm]{Hypothesis}
\newtheorem{hyp}{Hypothesis}
\theoremstyle{definition}
\newtheorem{dfn}[thm]{Definition}
\theoremstyle{remark}
\newtheorem{remark}[thm]{Remark}
\newtheorem{ex}[thm]{Example}
\newtheorem{q}[thm]{Question}

\hypersetup{pdftitle={Measure Theory}}
\hypersetup{pdfauthor={Josephine Evans }}


\author{
Josephine Evans
}
\title{Measure Theory: Assignment Two - Constructing new measurable functions from old measurable functions}
%\date{}

\makeindex

\begin{document}
\section{Measure Theory: Assignment Two - Constructing new measurable functions from old measurable functions}This sheet is all about building new measurable functions from existing measurable functions. Throughout this sheet we consider functions taking values in $\mathbb{R}$. We are interested in their measurability with respect to the Borel $\sigma$-algebra.
\begin{q}
Suppose that $f$ is a measurable function. Show that $-f$ and $\lambda f$ are both measurable functions, where $\lambda$ is a strictly positive contant. \emph{2 marks}
\end{q}
%\begin{ans}
%$(-f)^{-1}((-\infty, b]) = f^{-1}([-b, \infty))$ which is measurable as $[-b, \infty)$ is a Borel set. Similarly, $(\lambda f)^{-1}((-\infty, b]) = f^{-1}((-\infty, b/\lambda])$ which is also measurable.
%\end{ans}


\begin{q}
Suppose that $f_1$ and $f_2$ are measurable. Show that $f_1 \wedge f_2 = \min\{f_1, f_2\}$ is also measurable. \emph{3 marks}
\end{q}
%\begin{ans}
%We look at the set $\{ x \,:\, \min\{ f_1(x), f_2(x)\} \leq b\}$ this is equal to $\{ x \,:\, f_1(x) \leq b\} \cap \{ x \,:\, f_2(x) \leq b\}$. Since both $f_1$ and $f_2$ are measurable this set will be measurable. Therefore $\min\{f_1, f_2\}$ is measurable.
%\end{ans}

\begin{q}
Suppose that $f_1, f_2, f_3, \dots$ is a sequence of measurable functions. Show that $\inf_n f_n$ is also a measurable function. Use this to show that $\sup_n f_n$ is a measurable function as well. \emph{5 marks}
\end{q}
%\begin{ans}
%let $f(x) = \inf\{f_n(x)\}$ then we have $f^{-1}((-\infty, b]) = \{ x \,:\, f(x) \leq b\} = \{ x \, :\, f_n(x) \leq b \, \forall \, n\} = \bigcap_n \{ x \,:\, f_n(x) \leq b\}$. Since $f_n$ is measurable for every $n$ $f^{-1}((-\infty,b])$ is the countable intersection of measurable sets so measurable. The corresponding result with supremums follows from the fact that $\sup_n \{ f_n \} = - \inf_n \{ -f_n\}$.
%\end{ans}

\begin{q}
Again let $f_1, f_2, f_3, \dots$ be a sequence of measurable functions. Show that $\limsup_n f_n$ and $\liminf_n f_n$ are both measurable functions. Why does this mean that $\lim_n f_n$ will be a measurable function if it exists. \emph{5 marks}
\end{q}
%\begin{ans}
%Let us write $f(x) = \liminf_n f_n(x)$. Then $f^{-1}((-\infty, b)) = \{ x \,:\, \liminf_n f_n (x) < b\} = \{x \, :\, \exists m \, \mbox{s.t} \, \inf_{n \geq m} f_n(x) < b\} = \bigcup_m \bigcap_{n \geq m} \{ x \, :\, f_n (x) < b\}$. Therefore we have written this set as a countable union of a countable intersection of measurable sets, so it is measurable. Note that in order to use inf we had to work with $(-\infty, b)$. 
%\end{ans}

\begin{q}
Suppose that $f$ is a measurable function, show that $f^2$ is measurable. \emph{2 marks}
\end{q}
%\begin{ans}
%The set $\{ x \,:\, f(x)^2 \leq b\} = \{ x \, :\, -b \leq f(x) \leq b\}$ which is measurable. 
%\end{ans}

\begin{q}
Using the previous question, or otherwise, show that if $f_1$ an $f_2$ are measurable the $f_1 f_2$ is also measurable.  \emph{3 marks}
\end{q}
%\begin{ans}
%By the previous part $f_1^2, f_2^2$ and $(f_1+f_2)^2$ are all measurable. Therfore $f_1f_2 = \frac{1}{2}((f_1+f_2)^2 -f_1^2-f_2^2)$ is measurable.
%\end{ans}

\begin{q}
Let $f$ be a measurable function with $f >0$ everywhere. Show that $1/f$ is also measurable. \emph{5 marks}
\end{q}

%\begin{ans}
%$(1/f)^{-1}((-\infty, b]) = \{ x \,:\, 1/f(x) \leq b\} = \{ x \, :\, f(x) \geq 1/b \} = f^{-1}([b, \infty)}$ so this set is a measurable set therefore $1/f$ is measurable.
%\end{ans}

\end{document}