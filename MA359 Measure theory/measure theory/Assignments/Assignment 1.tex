\documentclass[11pt]{article}

\usepackage{mathtools}
\usepackage{amsmath, amsthm, amsfonts,amssymb}
\usepackage{enumitem}
\usepackage{graphicx}
\usepackage{colortbl}
\usepackage{tikz}
\usepackage[utf8]{inputenc}
\usepackage{esint}
\usepackage{mathrsfs}
\usepackage{subfig,float}
\usepackage[T1]{fontenc}
\usepackage{mathrsfs}  
\usepackage{bbm} 
\usepackage{enumitem}
\usepackage{enumerate}
\usepackage{mathtools}
\usepackage{dsfont}
\newcommand{\set}[1]{\left\{#1\right\}}

\def\grad{\nabla}
\DeclareMathOperator{\dive}{div}
\DeclareMathOperator{\supp}{supp}
\DeclareMathOperator{\essup}{ess\,sup}
\DeclareMathOperator{\Lip}{Lip}
\DeclareMathOperator{\sgn}{sgn}

% Notation for differentials
\def\d{\,\mathrm{d}}
\def\dv{\d v}
\def \ddt{\frac{\mathrm{d}}{\mathrm{d}t}}
\def \ddt{\frac{\mathrm{d}}{\mathrm{d}t}}
\def \ddr{\frac{\mathrm{d}}{\mathrm{d}r}}
\newcommand{\sign}{\text{sign}}
\DeclareMathOperator{\hess}{Hess}

% Generate a PDF with hyperlinks in references.
\usepackage[colorlinks=true,linkcolor=blue,citecolor=blue,urlcolor=blue,breaklinks]{hyperref}

% Bibliography
%-----------------------------------------------------------------

% This uses a bibliography style which hyperlinks the paper titles to
% the paper URL specified in the bibtex file. It also uses natbib,
% which cites papers by name such as Euler (1770) instead of [17].
\usepackage{hyperref}
\usepackage{breakurl}
\usepackage[square,sort,comma,numbers]{natbib}
%\usepackage{natbib}
\usepackage{url}
%\bibliographystyle{plainnat-linked}
\bibliographystyle{plain}

%\usepackage[notcite,notref]{showkeys}
%\usepackage{hyperref}
%\usepackage{breakurl}
\usepackage[square,sort,comma,numbers]{natbib}
%\usepackage{natbib}
%\usepackage{url}
%\usepackage[colorlinks=blue,linkcolor=blue,citecolor=blue,urlcolor=blue,breaklinks]{hyperref}
\bibliographystyle{plain}
\DeclarePairedDelimiter\abs{\lvert}{\rvert}%
\DeclarePairedDelimiter\norm{\lVert}{\rVert}%


\addtolength{\oddsidemargin}{-.875in}
\addtolength{\evensidemargin}{-.875in}
\addtolength{\textwidth}{1.75in}

\addtolength{\topmargin}{-.875in}
\addtolength{\textheight}{1.75in}

% Shortcuts
%-----------------------------------------------------------------

%\newcommand{\abs}[1]{\left\mid#1\right\mid}
%\newcommand{\ap}[1]{\left\langle#1\right\rangle}
%\newcommand{\norm}[1]{\left\mid#1\right\mid}
% \newcommand{\tnorm}[1]{\left\mid\!\left\mid\!\left\mid#1\right\mid\!\right\mid\!\right\mid}

\definecolor{lpink}{rgb}{0.96, 0.76, 0.76}
\definecolor{dpink}{rgb}{0.97, 0.51, 0.47}
\definecolor{sky}{rgb}{0.53, 0.81, 0.92}
\definecolor{salmon}{rgb}{1.0, 0.55, 0.41}
\definecolor{orman}{rgb}{0.24, 0.7, 0.44}
\definecolor{aciksari}{rgb}{0.91, 0.84, 0.42}
\definecolor{dgrey}{rgb}{0.52, 0.52, 0.51}

\def\R{\mathbb{R}}
\def\C{\mathbb{C}}
\def\P{\mathscr{P}}
\def\NN{\mathbb{N}}
\def\Q{\mathbb{Q}}

\def\ird{\int_{\mathbb{R}^N}}
\def\d{\,\mathrm{d}}
\def\dx{\,\mathrm{d}x}
\def\dy{\,\mathrm{d}y}
\def\p{\,\partial}
\newcommand{\en}{\mathcal{H}}
\newcommand{\havva}[1]{{\textcolor{blue}{[\textbf{H:} #1]}}}

% Operators

\def\grad{\nabla}
\def\weakto{\rightharpoonup}

\DeclareMathOperator{\divergence}{div}
%\newcommand{\dv}[1]{\divergence \left(#1\right)}
%
%\DeclareMathOperator{\supp}{supp}
%\DeclareMathOperator{\essup}{ess\,sup}
%\DeclareMathOperator{\Lip}{Lip}
\DeclareMathOperator{\law}{law}

\DeclareMathOperator{\Pf}{Pf.}
\DeclareMathOperator{\Fp}{Fp.}
\DeclareMathOperator{\pv}{pv.}

\newcommand{\for}{\quad \text{ for all }}
\newcommand{\tb}[1]{\textcolor{blue}{#1}}


\newcommand{\OmT}{\Omega\times (0,T)}
\let\pa\partial
\let\na\nabla
\newcommand{\red}[1]{\textcolor{red}{#1}}
\newcommand{\cD}{\mathcal{D}}
\let\eps\varepsilon
\newcommand{\wen}{w^{(\varepsilon,N)}}
\newcommand{\OmTc}{\overline{\Omega}\times [0,T]}
\newcommand{\rrhoA}{\sqrt{\rho_A}}
\newcommand{\rrhoB}{\sqrt{\rho_B}}
\newcommand{\rrhoAB}{\sqrt{\rho_A\rho_B}}

% Theorems
%-----------------------------------------------------------------
\newtheorem{thm}{Theorem}[section]
% \newtheorem{thm}{Theorem}
\newtheorem{cor}[thm]{Corollary}
\newtheorem{lem}[thm]{Lemma}
\newtheorem{prp}[thm]{Proposition}
\newtheorem{claim}[thm]{Claim}
% \newtheorem{hyp}[thm]{Hypothesis}
\newtheorem{hyp}{Hypothesis}
\theoremstyle{definition}
\newtheorem{dfn}[thm]{Definition}
\theoremstyle{remark}
\newtheorem{remark}[thm]{Remark}
\newtheorem{ex}[thm]{Example}
\newtheorem{q}[thm]{Question}

\hypersetup{pdftitle={Measure Theory}}
\hypersetup{pdfauthor={Josephine Evans }}


\author{
Josephine Evans
}
\title{Measure Theory: Assignment One - Lebesgue measure in $\mathbb{R}^d$}
%\date{}

\makeindex

\begin{document}
	\section{Assignment One - Lebesgue measure in $\mathbb{R}^d$}
This assignment is about defining Lebesgue measure on $\mathbb{R}^d$ as opposed to $\mathbb{R}$. Later in the course we will also see product $\sigma$-algebras and product measures which give another way of doing this. We begin by defining Lebesgue outer measure, $\lambda^*$ on $\mathbb{R}^d$ by first defining the measure of the rectangle $(a_1,b_1] \times (a_2, b_2] \times \dots \times (a_d, b_d]$. We define
\[ \lambda \left( (a_1,b_1] \times (a_2, b_2] \times \dots \times (a_d, b_d]\right)  = (b_1-a_1)(b_2-a_2)\dots(b_d-a_d).\] Then for any subset of $\mathbb{R}^d$, $A$, we define
\[ \lambda^*(A) = \inf\{ \sum_{n=1}^\infty \lambda(R) \,:\, \mbox{$R_k$ are rectangles}, A \subseteq \bigcup_{n=1}^\infty R_n \}.  \]
\begin{q}
Show that $\lambda^*$ is indeed an outer measure. \emph{3 marks}
\end{q}

\begin{q}
Show that if $R$ is a rectangle then $\lambda(R) = \lambda^*(R)$. \emph{5 marks}
\end{q}

We now recall that a set, $A$, will be $\lambda^*$- measureable if for every set $B$
\[ \lambda^*(B) = \lambda^*(B \cap A) + \lambda^*(B \cap A^c). \] As for the one dimensional case we write $\mathcal{M}$ for the set of Lebesgue measurable sets. We know from the proof of Carath\'eodory's extension theorem that $\mathcal{M}$ is a $\sigma$-algebra

\begin{q}
Explain why if $\lambda^*(A) =0$ then $\lambda^*(B \cap A)$ will also be zero. Therefore show that if $\lambda^*(A) =0$ or $\lambda^*(A^c) =0$ then $A$ is $\lambda^*$-measureable.  \emph{5 marks}
\end{q}

\begin{q}
In this question we will prove that every Borel subset of $\mathbb{R}^d$ is Lebesgue measurable.
\begin{itemize}
	\item Show that every half space of the form $H_{j,b} = \{ (x_1, \dots, x_d) \,:\, x_j \leq b \}$ is Lebesgue measurebale. \emph{5 marks}
	\item Show that every rectangle $R$ is Lebesgue measurable. \emph{2 marks}
	\item Show that every open set in $\mathbb{R}^d$ is a countable union of rectangles. \emph{3 marks}
	\item Show that every Borel set is Lebesgue measureable. \emph{2 marks}
\end{itemize}
\end{q}
Again we define Lebesgue measure on $\mathbb{R}^d$ to be the restriction $\lambda^*$ to $\mathcal{M}$.

%\begin{q}
%Use the uniqueness of extension theorem from the notes to show that Lebesgue measure is translation invariant. That is to say if we define the set $A+x = \{z \in \mathbb{R}^d\,:\, z=x+y, y \in A \}$. Show also that Lebesgue measure in $\mathbb{R}^d$ is rotationally invariant. That is to say if $M$ is a rotation matrix in $\mathbb{R}^d$ then $\lambda(A) = \lambda(M^{-1}(A))$ where we understand $M$ here to represent the map $x \mapsto Mx$.
%\end{q}
%
%\begin{q}
%Let $\lambda$ be Lebesgue measure on $\mathbb{R}^d$ let $L = \{ (x,y)\,:\, y=mx+c \}$ for some $m, c \in \mathbb{R}$. Show that $\lambda{L}=0$.
%\end{q}
%
%\begin{q}
%Let $f$ be the map on $\mathbb{R}^d$ given by $f(x) = 3x$. Write down an expression for $\lambda(f^{-1}(A))$ and prove that it is correct. 
%\end{q}


\end{document}