\documentclass[11pt]{article}

\usepackage{mathtools}
\usepackage{amsmath, amsthm, amsfonts,amssymb}
\usepackage{enumitem}
\usepackage{graphicx}
\usepackage{colortbl}
\usepackage{tikz}
\usepackage[utf8]{inputenc}
\usepackage{esint}
\usepackage{mathrsfs}
\usepackage{subfig,float}
\usepackage[T1]{fontenc}
\usepackage{mathrsfs}  
\usepackage{bbm} 
\usepackage{enumitem}
\usepackage{enumerate}
\usepackage{mathtools}
\usepackage{dsfont}
\newcommand{\set}[1]{\left\{#1\right\}}

\def\grad{\nabla}
\DeclareMathOperator{\dive}{div}
\DeclareMathOperator{\supp}{supp}
\DeclareMathOperator{\essup}{ess\,sup}
\DeclareMathOperator{\Lip}{Lip}
\DeclareMathOperator{\sgn}{sgn}

% Notation for differentials
\def\d{\,\mathrm{d}}
\def\dv{\d v}
\def \ddt{\frac{\mathrm{d}}{\mathrm{d}t}}
\def \ddt{\frac{\mathrm{d}}{\mathrm{d}t}}
\def \ddr{\frac{\mathrm{d}}{\mathrm{d}r}}
\newcommand{\sign}{\text{sign}}
\DeclareMathOperator{\hess}{Hess}

% Generate a PDF with hyperlinks in references.
\usepackage[colorlinks=true,linkcolor=blue,citecolor=blue,urlcolor=blue,breaklinks]{hyperref}

% Bibliography
%-----------------------------------------------------------------

% This uses a bibliography style which hyperlinks the paper titles to
% the paper URL specified in the bibtex file. It also uses natbib,
% which cites papers by name such as Euler (1770) instead of [17].
\usepackage{hyperref}
\usepackage{breakurl}
\usepackage[square,sort,comma,numbers]{natbib}
%\usepackage{natbib}
\usepackage{url}
%\bibliographystyle{plainnat-linked}
\bibliographystyle{plain}

%\usepackage[notcite,notref]{showkeys}
%\usepackage{hyperref}
%\usepackage{breakurl}
\usepackage[square,sort,comma,numbers]{natbib}
%\usepackage{natbib}
%\usepackage{url}
%\usepackage[colorlinks=blue,linkcolor=blue,citecolor=blue,urlcolor=blue,breaklinks]{hyperref}
\bibliographystyle{plain}
\DeclarePairedDelimiter\abs{\lvert}{\rvert}%
\DeclarePairedDelimiter\norm{\lVert}{\rVert}%


\addtolength{\oddsidemargin}{-.875in}
\addtolength{\evensidemargin}{-.875in}
\addtolength{\textwidth}{1.75in}

\addtolength{\topmargin}{-.875in}
\addtolength{\textheight}{1.75in}

% Shortcuts
%-----------------------------------------------------------------

%\newcommand{\abs}[1]{\left\mid#1\right\mid}
%\newcommand{\ap}[1]{\left\langle#1\right\rangle}
%\newcommand{\norm}[1]{\left\mid#1\right\mid}
% \newcommand{\tnorm}[1]{\left\mid\!\left\mid\!\left\mid#1\right\mid\!\right\mid\!\right\mid}

\definecolor{lpink}{rgb}{0.96, 0.76, 0.76}
\definecolor{dpink}{rgb}{0.97, 0.51, 0.47}
\definecolor{sky}{rgb}{0.53, 0.81, 0.92}
\definecolor{salmon}{rgb}{1.0, 0.55, 0.41}
\definecolor{orman}{rgb}{0.24, 0.7, 0.44}
\definecolor{aciksari}{rgb}{0.91, 0.84, 0.42}
\definecolor{dgrey}{rgb}{0.52, 0.52, 0.51}

\def\R{\mathbb{R}}
\def\C{\mathbb{C}}
\def\P{\mathscr{P}}
\def\NN{\mathbb{N}}
\def\Q{\mathbb{Q}}

\def\ird{\int_{\mathbb{R}^N}}
\def\d{\,\mathrm{d}}
\def\dx{\,\mathrm{d}x}
\def\dy{\,\mathrm{d}y}
\def\p{\,\partial}
\newcommand{\en}{\mathcal{H}}
\newcommand{\havva}[1]{{\textcolor{blue}{[\textbf{H:} #1]}}}

% Operators

\def\grad{\nabla}
\def\weakto{\rightharpoonup}

\DeclareMathOperator{\divergence}{div}
%\newcommand{\dv}[1]{\divergence \left(#1\right)}
%
%\DeclareMathOperator{\supp}{supp}
%\DeclareMathOperator{\essup}{ess\,sup}
%\DeclareMathOperator{\Lip}{Lip}
\DeclareMathOperator{\law}{law}

\DeclareMathOperator{\Pf}{Pf.}
\DeclareMathOperator{\Fp}{Fp.}
\DeclareMathOperator{\pv}{pv.}

\newcommand{\for}{\quad \text{ for all }}
\newcommand{\tb}[1]{\textcolor{blue}{#1}}


\newcommand{\OmT}{\Omega\times (0,T)}
\let\pa\partial
\let\na\nabla
\newcommand{\red}[1]{\textcolor{red}{#1}}
\newcommand{\cD}{\mathcal{D}}
\let\eps\varepsilon
\newcommand{\wen}{w^{(\varepsilon,N)}}
\newcommand{\OmTc}{\overline{\Omega}\times [0,T]}
\newcommand{\rrhoA}{\sqrt{\rho_A}}
\newcommand{\rrhoB}{\sqrt{\rho_B}}
\newcommand{\rrhoAB}{\sqrt{\rho_A\rho_B}}

% Theorems
%-----------------------------------------------------------------
\newtheorem{thm}{Theorem}[section]
% \newtheorem{thm}{Theorem}
\newtheorem{cor}[thm]{Corollary}
\newtheorem{lem}[thm]{Lemma}
\newtheorem{prp}[thm]{Proposition}
\newtheorem{claim}[thm]{Claim}
% \newtheorem{hyp}[thm]{Hypothesis}
\newtheorem{hyp}{Hypothesis}
\theoremstyle{definition}
\newtheorem{dfn}[thm]{Definition}
\theoremstyle{remark}
\newtheorem{remark}[thm]{Remark}
\newtheorem{ex}[thm]{Example}

\hypersetup{pdftitle={Measure Theory}}
\hypersetup{pdfauthor={Josephine Evans }}


\author{
Josephine Evans
}
\title{Measure Theory}
%\date{}

\makeindex

\begin{document}

\section{Outer measure, Lebesgue Measure}

\begin{dfn}[Outer measure]
We write $\mathscr{P}(E)$ to be the power set of $E$, that is to say the set of all subsets of $E$. An outer measure is a function, $\nu$, from $\mathscr{P}(E) \rightarrow \mathbb{R}_+ \cup \{\infty\}$ such that
\begin{itemize}
\item $\nu(\emptyset) =0$,
\item If $A \subseteq B$ then $\nu (A) \leq \nu(B)$, (this is called \emph{monotonicity})
\item If $A_1, A_2, \dots$ is a sequence of subsets then $\nu\left( \bigcup_n A_n \right) \leq \sum_n \nu(A_n)$, (this is called \emph{countable subadditivity}).
\end{itemize}
\end{dfn}

The key example of an outer measure is \emph{Lebesgue outer measure}, defining this is our first step to defining Lebesgue measure.
\begin{dfn}[Lebesgue measure on unions of intervals]
Let us call $\mathcal{I}$ the set of countable unions of half-open intervals $(a,b]$. That is to say $I$ is the set of all sets of the form
\[ \bigcup_n (a_1,b_1]. \]  We also write $\mathcal{J}$ to be the set of finite unions of half open intervals. Then we define a set function $\lambda$ from $\mathcal{J}$ to $\mathbb{R}$ by
\[ \lambda \left(  (a_1, b_1] \cup (a_2,b_2] \cup \dots \cup (a_n,b_n] \right) = \sum_{i=1}^n (b_i-a_i).\]
\end{dfn}
We are most interested in $\lambda$ defined on a single half open interval. 
Using this we can define Lebesgue outer measure.
\begin{dfn}[Lebesgue outer measure]
We define Lebesgue outer measure on $\mathscr{P}(\mathbb{R})$ by 
\[ \lambda^* (A) = \inf \{ \sum_n\lambda (I_n) \, : \, \mbox{$I_n$ are half open intervals}\, , A \subset \bigcup_n I_n\}. \]
\end{dfn}

\begin{prp}
Lebesgue outer measure is an outer measure and agrees with $\lambda$ on any half open interval.
\end{prp}
\begin{proof}
We need to check each part of the definition of outer measure. First the fact that $\lambda^*(\emptyset) = 0$ follows from the fact that $\emptyset \in \mathcal{I}$ and $\lambda(\emptyset) = 0$. Now suppose that $A_1 \subset A_2$, then any set $B \in \mathcal{I}$ with $A_2 \subseteq B$ also has $A_1 \subseteq B$ so 
\[ \inf \{ \sum_n\lambda (I_n) \, : \, \mbox{$I_n$ are intervals}\,, A_1 \subset B = \bigcup_n I_n\} \leq \inf \{ \sum_n\lambda (I_n) \, : \, \mbox{$I_n$ are intervals}\,, A_2 \subset B = \bigcup_n I_n\}, \] as the infimum over a larger set will always be smaller. Now let us turn to the countable subadditivity. Let us take some sequence $A_1, A_2, \dots$, if $\sum_n \lambda^*(A_n) = \infty$ then we are done. Therefore we can assume that $\sum_n \lambda^*(A_n) < \infty$. Now let us fix an arbitrary $\epsilon>0$. Now by the definition of $\lambda^*$ for each $n$ there exists some $I_n \in \mathcal{I}$ such that $A_n \subseteq I_n$ and $I_n = \bigcup_k I_{n,k}$ where the $I_{n,k}$ are half open intervals, and $\sum_k\lambda(I_{n,k}) \leq \lambda^* (A_n) + \epsilon 2^{-n}$. Then the set $I = \bigcup_n I_n$ is in $\mathcal{I}$ and $\sum_{n,k}\lambda(I_{n,k}) = \sum_n \sum_k\lambda(I_{n,k}) \leq \sum_n \lambda^*(A_n) + \epsilon$. Therefore $\lambda^*(\bigcup_n A_n) \leq  \sum_n \lambda^*(A_n) + \epsilon$.

Lastly if $A$ is the interval $(a,b]$ then $(a,b] \in \mathcal{I}$ so $\lambda^*(A) \leq b-a$. Suppose that $(a,b] \subseteq (c_1,d_1] \cup (c_2, d_2] \cup \dots$. Without loss of generality rewrite our intervals so that $d_i \neq c_j$ for any $j \neq i$ (by sticking two intervals together) and this does not change the value of $\sum_n (d_n - c_n)$. For any fixed epsilon we have $[a+\epsilon,b-\epsilon] \subseteq (c_1,d_1) \cup (c_2,d_2) \cup \dots$. This is because we can take interiors of both sides and the interior of $(c_1, d_1] \cup, (c_2, d_2] \cup \dots = (c_1,d_1) \cup (c_2, d_2) \cup \dots$ whenever the $\bigcup_n \partial ((c_n, d_n]) \subseteq \partial \left( \bigcup_n (c_n,d_n]\right)$ where $\partial$ indicates the boundary of a set. By compactness of closed intervals there is some $n$ so that $[a+\epsilon, b- \epsilon] \subseteq (c_1,d_1) \cup \dots (c_n, d_n)$. We can then compare volumes safely and get
\[ b-a -2\epsilon \leq d_1-c_1 + \dots + d_n -c_n \leq \sum_n (d_n - c_n). \] Letting $\epsilon$ go to $0$ gives
\[ b-a \leq \sum_n (d_n-c_n). \] Now ranging over all possible covering sequences gives
\[ b-a \leq \lambda^*((a,b]). \]

Alternative proof that $\lambda$ and $\lambda^*$ agree on intervals by enlarging the intervals on RHS. Suppose that $(a,b] \subseteq (c_1,d_1] \cup (c_2, d_2] \cup \dots$.  Then we have that for any $\epsilon, \delta$ that 
\[ [a+\epsilon, b-\epsilon] \subseteq (c_1-\delta/2, d_1 + \delta/2) \cup (c_2 - \delta/4, d_2 + \delta/4) \cup \dots (c_k -2^{-k} \delta, d_k + 2^{-k} \delta) \cup \dots.  \] Then using compactness there exists some $n$ such that
\[ [a+\epsilon, b-\epsilon] \subseteq \bigcup_{k=1}^n (c_k -2^{-k}\delta, d_k + 2^{-k} \delta). \] Then we can compare the lengths of these two sets to get
\[ b-a - 2\epsilon \leq \sum_{k=1}^n (d_k - c_k + 2^{-k+1} \delta) \leq \sum_{k=1}^\infty (d_k-c_k) + 2 \delta. \] Both $\epsilon$ and $\delta$ are arbitrary so we can let them go to $0$ and get
\[ b-a \leq \sum_k (d_k - c_k). \] We then continue as above.

Note: When we are working in $\mathbb{R}^d$ as in the assignment the terms involving $\epsilon$ and $\delta$ will be multiplied by something involving the side lengths of rectangles. In order to run the proof you can say that wlog all the rectangles you are looking at are contained inside some fixed large rectangle. This will allow you to send $\epsilon$ and $\delta$ to zero without having to worry.
\end{proof}


We want to turn this outer measure into a true measure. In order to do this we need to restrict $\lambda^*$ to some subset of $\mathscr{P}(\mathbb{R})$.
\begin{dfn}[Lebesgue Measurable sets]
We call a set $A \in \mathscr{P}(\mathbb{R})$ is \emph{Lebesgue Measureable} if for any $B \in \mathscr{P}(\mathbb{R})$ we have \[ \lambda^*(B) = \lambda^*(A \cap B) + \lambda^*(A^c \cap B). \]
\end{dfn}

\begin{prp}
The collection of Lebesgue measureable sets, $\mathscr{M}$, is a $\sigma$ algebra.
\end{prp}
\begin{proof}
First let us notice that the definition of a Lebesgue measureable sets is symmetric in $A$ and $A^c$, so $A \in \mathscr{M}$ implies that $A^c \in \mathscr{M}$.

Secondly we can see that $\emptyset \in \mathscr{M}$ as $\lambda^*(A\cap \emptyset) + \lambda^*(A \cap \emptyset^c) = \lambda^*(\emptyset) + \lambda^*(A \cap E) = 0+ \lambda^*(A)$. This also implies via the first point that $E \in \mathscr{M}$.

We then show that if $A_1, A_2 \in \mathscr{M}$ then $A_1 \cup A_2 \in \mathscr{M}$. Using the fact that $A_1 \in \mathscr{M}$ we have
\[ \lambda^*(B \cap (A_1 \cup A_2)) = \lambda^*(B \cap (A_1 \cup A_2) \cap A_1) + \lambda^*(B \cap(A_1 \cup A_2) \cap A_1^c) = \lambda^* (B \cap A_1) + \lambda^* (B \cap A_2 \cap A_1^c).  \] We also have the identity $(A_1 \cup A_2)^c = A_1^c \cap A_2^c$ therefore
\[ \lambda^*(B \cap (A_1 \cup A_2)) + \lambda^*(B \cap (A_1 \cup A_2)^c) = \lambda^*(B \cap A_1) + \lambda^* (B \cap A_2 \cap A_1^c) + \lambda^*(B \cap A_1^c \cap A_2^c). \] Then since $A_2 \in \mathscr{M}$ we have
\[  \lambda^* (B \cap A_2 \cap A_1^c) + \lambda^*(B \cap A_1^c \cap A_2^c) = \lambda^*(B \cap A_1^c).\] Therefore,
\[   \lambda^*(B \cap (A_1 \cup A_2)) + \lambda^*(B \cap (A_1 \cup A_2)^c) = \lambda^*(B \cap A_1) + \lambda^*(B \cap A_1^c). \] Then we use again the fact that $A_1 \in \mathscr{M}$ to get
\[  \lambda^*(B \cap (A_1 \cup A_2)) + \lambda^*(B \cap (A_1 \cup A_2)^c) =\lambda^*(B). \] This shows that $A_1 \cup A_2 \in \mathscr{M}$. 

Now let us take an infinite sequence of disjoint sets $A_1, A_2, A_3, \dots$ then we will show 
\[ \lambda^*(B) = \sum_{i=1}^n \lambda^* (B \cap A_i) + \lambda^*\left( B \cap \left( \bigcap_{i=1}^n A_i^c \right) \right). \] We can show this by induction. For the base case it just follows with $n=1$ from the fact that $A_1 \in \mathscr{M}$. Then by induction suppose we know that
\[ \lambda^*(B) = \sum_{i=1}^{n-1} \lambda^*(B \cap A_i) + \lambda^*\left( B \cap \left( \bigcap_{i=1}^{n-1} A_i^c \right) \right).  \] Now since $A_n \in \mathscr{M}$ we have
\[ \lambda^*\left( B \cap \left( \bigcap_{i=1}^{n-1} A_i^c \right) \right) = \lambda^* \left(B \cap A_n \cap \left( \bigcap_{i=1}^{n-1} A_i^c \right)  \right) + \lambda^* \left( B \cap A_n^c \cap \left( \bigcap_{i=1}^{n-1} A_i^c \right) \right).\] Now since $A_n$ is disjoint from $A_1, \dots, A_{n-1}$ we have that $A_n \cap \left( \bigcap_{i=1}^{n-1} A_i^c \right)  = A_n$ so we have \[  \lambda^*\left( B \cap \left( \bigcap_{i=1}^{n-1} A_i^c \right) \right) = \lambda^*(B \cap A_n) + \lambda^*\left( B \cap \left( \bigcap_{i=1}^{n} A_i^c \right) \right). \] This gives our induction step. 

By monotonicity of the outer measure this gives that for any $n$ we have
\[ \lambda^*(B) \geq \sum_{i=1}^n \lambda^*(B \cap A_i) + \lambda^* \left(B \cap \left( \bigcap_{i=1}^\infty A_i^c \right) \right). \] Consequently we can let $n$ tend to infinity to get
\[ \lambda^*(B) \geq \sum_{i=1}^\infty \lambda^*(B \cap A_i) +\lambda^* \left(B \cap \left( \bigcap_{i=1}^\infty A_i^c \right) \right). \] Now we can use the countable subadditivity of $\lambda^*$ to get
\[ \lambda^*(B) \geq \lambda^*\left( B \cap \left( \bigcup_{i=1}^\infty A_i \right)\right) + \lambda^*\left((B \cap \left( \bigcap_{i=1}^\infty A_i^c \right) \right) = \lambda^* \left( B \cap \left( \bigcup_{i=1}^\infty A_i \right) \right) + \lambda^* \left( B \cap \left( \bigcup_{i=1}^\infty A_i \right)^c \right).\] Furthermore, the subadditivity of $\lambda^*$ gives
\[ \lambda^*(B) \leq  \lambda^* \left( B \cap \left( \bigcup_{i=1}^\infty A_i \right) \right) + \lambda^* \left( B \cap \left( \bigcup_{i=1}^\infty A_i \right)^c \right). \] Therefore, 
\[ \lambda^*(B) =  \lambda^* \left( B \cap \left( \bigcup_{i=1}^\infty A_i \right) \right) + \lambda^* \left( B \cap \left( \bigcup_{i=1}^\infty A_i \right)^c \right). \]

We have now shown that $\mathscr{M}$ is closed under complements and taking countable unions and contains $\emptyset$ which is sufficient to show that $\mathscr{M}$ is a $\sigma$-algebra.
\end{proof}

\begin{prp}
The restriction of $\lambda^*$ to $\mathscr{M}$ is a measure. 
\end{prp}
\begin{proof}
We need to show that $\lambda^*$ is countably additive on $\mathscr{M}$ so let $A_1, A_2, \dots$ be a sequence of disjoint subsets in $\mathscr{M}$. In the proof that $\mathscr{M}$ is a $\sigma$-algebra we showed that 
\[ \lambda^*(B) \geq \sum_{i=1}^\infty \lambda^*(B \cap A_i) + \lambda^* \left( B \cap \left( \bigcap_{i=1}^\infty A_i^c \right) \right).  \] Now let us take the particular case where $B = \bigcup_{i=1}^\infty A_i$ this gives
\[ \lambda^* \left( \bigcup_{i=1}^\infty A_i \right) \geq \sum_{i=1}^n \lambda^*(A_i) + \lambda^* \left(\left( \bigcup_{i=1}^\infty A_i \right) \cap \left( \bigcup_{i=1}^\infty A_i \right)^c \right) = \sum_{i=1}^\infty \lambda^* (A_i). \] Countable subadditivity gives
\[ \lambda^* \left( \bigcup_{i=1}^\infty A_i \right) \leq \sum_{i=1}^n \lambda^*(A_i),  \] so consequently
\[ \lambda^* \left( \bigcup_{i=1}^\infty A_i \right) = \sum_{i=1}^n \lambda^*(A_i).  \]
\end{proof}

\begin{remark}
We now call the restriction of $\lambda^*$ to $\mathscr{M}$, $\lambda$ and call it \emph{Lebesgue measure}.
\end{remark}

We now want to know that there are some Lebesgue measureable sets. In order to do this we first show that all the intervals of the form $(-\infty, b]$ are Lebesgue measurable.

\begin{lem}
The intervals of the form $(-\infty, b]$ are Lebesgue measureable.
\end{lem}
\begin{proof}
Let $B$ be a subset of $\mathbb{R}$ and let $I_1, I_2, \dots$ be a sequence of half open intervals such that $B \subseteq I_1 \cup I_2 \cup \dots$. Now let us define the (often empty) intervals $I^l_i = I_i \cap (-\infty, b]$ and $I^r_i = I_i \cap (b, \infty)$, these are also half open intervals. We have $B \cap (-\infty,b] \subseteq \bigcup_n I^l_n$ and $B \cap (b,\infty) \subseteq \bigcup_n I^r_n$. Therefore we have
\[ \lambda^*(B \cap (-\infty, b]) \leq \sum_n \lambda(I^l_n), \quad \lambda^*(B \cap (b,\infty)) \leq \sum_n \lambda(I^r_n). \] Using this we have
\[ \lambda^*(B \cap(-\infty,b])) + \lambda^*(B \cap (b,\infty)) \leq \sum_n \lambda(I^l_n) + \sum_n \lambda(I^r_n) = \sum_n \lambda(I_n).\] We can then take the infimum over all possible sequences of intervals covering $B$ to get
\[ \lambda^*(B \cap(-\infty,b])) + \lambda^*(B \cap (b,\infty)) \leq \lambda^*(B). \] Combining this with countable subadditivity gives
\[ \lambda^*(B) = \lambda^*(B \cap(-\infty,b])) + \lambda^*(B \cap (b,\infty)). \] Therefore, $(-\infty,b]$ is Lebesgue measurable.
\end{proof}

\begin{cor}
Every set in $\mathcal{B}(\mathbb{R})$ is Lebesgue measurable.
\end{cor}
\begin{proof}
The Borel $\sigma$ algebra is the $\sigma$ algebra generated by sets of the form $(-\infty, b]$ as shown last week. Therefore, as $\mathscr{M}$ is a $\sigma$-algebra and contains all the intervals of the form $(-\infty, b]$ then it contains the Borel $\sigma$-algebra.
\end{proof}





The construction of Lebesgue measure via the outer measure can be generalised via Carath\'eodory's extension theorem. We briefly give the defintion of a ring of subsets.
\begin{dfn}[Ring]
A collection of subsets, $\mathcal{A}$, of a space $E$ is called a ring if for every $A,B \in \mathcal{A}$ we have $A \setminus B \in \mathcal{A}$ and $A \cup B \in \mathcal{A}$.
\end{dfn}
Now we introduce Carath\'eodory's Extension theorem. We can see that the proof is in many ways very similar to the construction of Lebesgue measure. 
\begin{thm}[Carath\'eodory's Extension Theorem]
Let $\mathcal{A}$ be a ring of subsets of $E$, and let $\mu: \mathcal{A} \rightarrow [0, \infty]$ be a countably additive set function. Then $\mu$ extends to a measure on $\sigma(\mathcal{A})$.
\end{thm}
\begin{proof}
We define the outer measure $\mu^*$ on $\mathscr{P}(E)$ by
\[ \mu^*(B) = \inf \left\{ \sum_n \mu(A_n) \,:\, A_n \in \mathcal{A} \forall n, B \subset \bigcup_n A_n \right\}. \] $\mu^*(B) = \infty$ if there is not possible sequence of $A_n$ so that $B$ is contained in their union. We can see immediately that $\mu^*(\emptyset) =0$ and $\mu^*$ is increasing. 

As before we define $\mathscr{M}$ to be the set of $\mu^*$ measurable sets $A$ that satisfy, for every $B \subseteq E$ that
\[ \mu^*(B) = \mu^*(B \cap A) + \mu^*(B \cap A^c).  \] We want to show that $\mathscr{M}$ is a $\sigma$-algebra and that $\mu^*$ restricts to a measure on $\mathscr{M}$.

First we show that $\mu^*$ is countably subadditive. Suppose that we have a sequence $B_n$ and want to show that
\[ \mu^*\left( \bigcup_n B_n\right) \leq \sum_n \mu^*(B_n). \] Let us fix some $\epsilon >0$ then for each $n$ there is a sequence $A_{n,m} \in \mathcal{A}$ such that $B_n \subset \bigcup_m A_{n,m}$ and $\sum_m \mu(A_{n,m}) \leq \mu^*(B_n) + \epsilon 2^{-n}$. Then $\bigcup_n B_n \subset \bigcup_{n,m} A_{n,m}$ and $\sum_{n,m}\mu(A_{n,m}) \leq \sum_n \mu^*(B_n) + \epsilon$. Therefore $\mu^* \left( \bigcup_n B_n \right) \leq \sum_n \mu(B_n) + \epsilon$. Since $\epsilon$ is arbitrary this gives the countable subadditivity. 

Now we show that $\mu^*$ agrees with $\mu$ on $\mathcal{A}$. Let us take $A \in \mathcal{A}$ clearly $A \subseteq A$ so $\mu^*(A) \leq \mu(A)$. Now suppose that there is a sequence $A_n \in \mathcal{A}$ such that $A \subseteq \bigcup_n A_n$. Then $A \cap A_n = A \setminus (A \setminus A_n) \in \mathcal{A}$. Therefore we use the countable subadditivity of $\mu$ on $\mathcal{A}$ to get
\[ \mu (A) \leq  \sum_n \mu(A_n \cap A) \leq \sum_n \mu(A_n).\] Taking the infimum over such sequences gives $\mu(A) \leq \mu^*(A)$. Therefore $\mu$ and $\mu^*$ agree on $\mathcal{A}$. 

Now we show that $\mathscr{M}$ contains $\mathcal{A}$. That is to say we want to show that if $A \in \mathcal{A}$ then for every $B$
\[ \mu^*(B) = \mu^*(B \cap A) + \mu^*(B \cap A^c). \] Using subadditivity of $\mu^*$ we have that $\mu^*(B) \leq \mu^*(B \cap A) + \mu^*(B \cap A^c)$. Therefore we want to show $\mu^*(B) \geq \mu^*(B \cap A) + \mu^*(B \cap A^c)$. Let $A_n$ be a sequence in $\mathcal{A}$ such that $\mu^*(B) \geq \sum_n \mu(A_n) - \epsilon, B \subseteq \bigcup_n A_n$, then we already know that $A \cap A_n$ will be in $\mathcal{A}$ we also have that $A^c \cap A_n = A_n \setminus (A \cap A_n) \in \mathcal{A}$. Therefore $\mu(B \cap A) \leq \sum_n \mu(A_n \cap A)$ and $\mu(B \cap A^c) \leq \sum_n \mu(A^c \cap A_n)$ and consequently
\[ \mu^*(B \cap A) + \mu^*(B \cap A^c) \leq \sum_n \left( \mu(A_n \cap A) + \mu(A_n \cap A^c) \right) = \sum_n \mu(A_n) \leq \mu^*(B) + \epsilon. \] As $\epsilon$ is arbitrary this gives the required result. 

The next step is to show that $\mathscr{M}$ is a $\sigma$-algebra. We start with the algebra part. $E$ and $\emptyset$ are in $\mathscr{M}$ as
\[ \mu^*(B) = \mu^*(B \cap E) + \mu^*(B \cap \emptyset), \]
 just because $B \cap E = B$ and $B \cap \emptyset = \emptyset$ and we know $\mu^*(\emptyset) =0$. We also can see that
\[ \mu^*(B) = \mu^*(B \cap A) + \mu^*(B \cap A^c) \] is symmetric in exchanging $A$ and $A^c$ so if $A \in \mathscr{M}$ then so is $A^c$. Now suppose $A_1, A_2 \in \mathscr{M}$. We notice that $(A_1 \cap A_2)^c \cap A_1 = (A_1^c \cup A_2^c) \cap A_1 = (A_1^c \cap A_1) \cup (A_2^c \cap A_1) = A_2^c \cap A^1$ and $ A_1^c = A_1^c \cap(A_1 \cap A_2)^c$. Using this and the fact that $A_1, A_2, A_1^c, A_2^c$ are in $\mathscr{M}$ we have

\begin{align*}\mbox{Using that} \, A_1 \in \mathscr{M} \quad \mu^*(B) &= \mu^*(B \cap A_1) + \mu^*(B \cap A_1^c)\\
\mbox{Using that}\, A_2 \in \mathscr{M} \quad &= \mu^*(B \cap (A_1 \cap A_2)) + \mu^*(B \cap A_1 \cap A_2^c) + \mu^*(B \cap A_1^c)\\
\mbox{Using our first identity} \quad &= \mu^*(B \cap (A_1 \cap A_2)) + \mu^*(B \cap (A_1 \cap A_2)^c \cap A_1) + \mu^*(B \cap A_1^c)\\
\mbox{Using our second identiy} \quad &= \mu^*(B \cap (A_1 \cap A_2)) + \mu^*(B \cap (A_1 \cap A_2)^c \cap A_1) + \mu^*(B \cap (A_1 \cap A_2)^c \cap A_1^c)\\
\mbox{Using the fact that}\, A_1 \in \mathscr{M} \quad &= \mu^*(B \cap (A_1 \cap A_2)) + \mu^*(B \cap (A_1 \cap A_2)^c).
\end{align*}
Now that we have shown that $\mathscr{M}$ contains finite unions we want to show it countains countable unions. Let $A_n$ be a sequence of disjoin sets in $\mathscr{M}$. Let us write $A = \bigcup_n A_n$. Then itterating our previous result we have for any $B, n$ that
\[ \mu^*(B) = \sum_{k=1}^n \mu^*(B \cap A_k) + \mu^*( B \cap A_1^c \cap \dots \cap A_n^c). \] Now as $A^c \subseteq A_1^c \cap A_2^c \dots \cap A_n^c$ for each $n$ we have $\mu^*(B \cap A^c) \leq \mu^*(B \cap A_1^c \cap \dots \cap A_n^c)$. Therefore for each $n$
\[ \mu^*(B) \geq \sum_{k=1}^n \mu^*(B \cap A_k) + \mu^*(B \cap A^c). \] Letting $n \rightarrow \infty$ we have
\[ \mu^*(B) \geq \sum_n \mu^*(B \cap A_n) + \mu^*(B \cap A^c). \] Now we use the countable subadditivity of $\mu^*$ and the fact that $B \cap A = \bigcup_n (B \cap A_n)$ to get
\[ \mu^*(B) \geq \mu^*(B \cap A) + \mu^*(B \cap A^c). \] As the other inequality holds by subadditivity of $\mu^*$ we have that $\mu^*(B) = \mu^*(B \cap A) + \mu^*(B \cap A^c)$ and hence $A \in \mathscr{M}$.

Lastly, we want to show that $\mu^*$ is a measure on $\mathscr{M}$. In order to do this we need to show that $\mu^*$ is countably additive on $\mathscr{M}$. In the last step we showed that for any $B$, and a sequence of disjoint sets $A_n$ in $\mathscr{M}$ with $A= \bigcup_n A_n$, that
\[ \mu^*(B) \geq \sum_n \mu^*(A_n \cap B) + \mu^*(B \cap A^c). \] If we apply this identity with $B=A$ and use the fact that $A_n \cap A = A_n$ we get
\[ \mu^*(A) \geq \sum_n \mu^*(A_n). \] Since we already know that $\mu^*$ is countably subadditive this is sufficient to show that $\mu^*$ is countably additive and hence a measure on $\mathscr{M}$.
\end{proof}

\begin{thm}[Uniqueness of Extension]
Let $\mu_1$ and $\mu_2$ be measures on $(E,\mathcal{E})$ with $\mu_1(E) = \mu_2(E) < \infty$. Suppose that $\mu_1 = \mu_2$ on $\mathcal{A}$ where $\mathcal{A}$ is a $\pi$-system generating $\mathcal{E}$, then $\mu_1 = \mu_2$ on $\mathcal{E}$.
\end{thm}
\begin{proof}
Let us consider $\mathcal{D} \subseteq \mathcal{E}$ defined as the measurable sets on which $\mu_1(A) = \mu_2(A)$. By hypothesis $E \in \mathcal{D}$ and $\mathcal{A} \subseteq \mathcal{D}$. We want to show that $\mathcal{D}$ is a $\sigma$-algebra and therefore $\mathcal{D} = \mathcal{E}$. Suppose that $A, B \in \mathcal{E}$ with $A \subseteq B$ then we have $\mu_i(A) + \mu_i (B \setminus A) = \mu_i(B) < \infty.$ This means that if $A$ and $B$ are in $\mathcal{D}$ then so is $A \setminus B$. Therefore, $\mathcal{D}$ is a $d$-system containing the $\pi$-system $\mathcal{A}$ so by Dynkin's lemma is equal to $\mathcal{E}$.
\end{proof}


\end{document}